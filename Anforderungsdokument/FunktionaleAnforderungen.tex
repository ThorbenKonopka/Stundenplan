% !TeX root = Anforderungsdokument.tex

\clearpage

\newcounter{freq}[subsection]
\newcommand\printfreqnr{\stepcounter{freq}FA-\the\value{subsection}\num[minimum-integer-digits=2]{\thefreq}}

\section{Funktionale Anforderungen}
Wir haben uns zusammengesetzt und uns Anforderungen überlegt. Die Anforderungen sollen in Form von Texten zuerst erklärt und dann mit Satzschablonen spezifiziert werden. 

\vspace{6pt}

Die spezifizierten Anforderungen sollen die folgenden Bedingungen erfüllen.
\begin{itemize}
	\item Eine Anforderung muss überprüfbar sein. Wenn eine Anforderung nicht selbst vollständig überprüfbar ist, müssen weitere Anforderungen als Kriterien definiert werden.
	\item Jede Anforderung muss eine eindeutige ID haben. 
	\begin{itemize}
		\item Die ID von Funktionalen Anforderungen fängt mit \textbf{FA-} an.
		\item Die ID von Nicht Funktionalen Anforderungen fängt mit \textbf{NFA-} an.
	\end{itemize}
	\item Jede Anforderung muss sich aus dem über der Tabelle stehenden Text ergeben.
\end{itemize}

Die Anforderungen wurden Schrittweise ermittelt und spezifiziert. Zuerst wurde ein Text geschrieben, dem Anforderungen entnommen werden können. Unter dem Text wurde eine Tabelle erstellt, die die aus dem Text ausgearbeiteten Anforderungen enthält.

\newpage

\subsection{Einloggen}
Wir finden, dass eine einfache Anmeldung mit Benutzername und Passwort völlig ausreicht. Es macht zudem keinen Sinn verschiedene Anmeldefenster für die Verschiedenen Benutzerarten zu erstellen. Es ist vorgesehen, dass anhand des Benutzernamens herausgefunden werden kann, was für ein Benutzer sich gerade angemeldet hat.

\vspace{12pt}

\begin{tabular} {|p{1,4cm}|p{10,7cm}|p{2,7cm}|}
	\hline
	ID & Anforderung & Kriterien \\
	\hline
	\printfreqnr
	& Wenn ein unangemeldeter Benutzer das System aufruft \textbf{MUSS} Das System den Benutzer dazu auffordern sich mit seinem Benutzernamen und seinem Password anzumelden.
	& - \\
	\hline
	\printfreqnr
	& Wenn ein unangemeldeter Benutzer einen richtigen Benutzernamen und das richtige Passwort eingegeben hat \textbf{MUSS} das System den Benutzer anmelden und weiterleiten.
	& -  \\
	\hline
	\printfreqnr
	& Wenn ein unangemeldeter Benutzer einen falschen Benutzernamen oder ein falsches Passwort eingegeben hat, \textbf{MUSS} das System den Benutzer darauf hinweisen, dass die Anmeldung nicht erfolgreich war.
	& - \\
	\hline
	\printfreqnr
	& Wenn eine Anmeldung nicht erfolgreich war, \textbf{DARF} das System \textbf{NICHT} anzeigen, welche Eingabe nicht korrekt war.
	& - \\
	\hline
\end{tabular}

\newpage

\subsubsection{Mögliche Änderungen}
Hier sind die möglichen Änderungen für die Anforderungen in diesem Unterkapitel.

\vspace{6pt}

\begin{tabular} {|p{4cm}|p{2,1cm}|p{8,7cm}|}
	\hline
	Anforderungen & Stabilität & Begründung \\
	\hline
	FA-101
	& 90\%
	& Dies ist eine Standardanforderung, lediglich zusätzliche Sicherungsmöglichkeiten, wie z.B. 2FA, sind denkbar. \\
	\hline
	FA-102
	& 100\%
	& Grundkriterium für das Funktionieren der Nutzerauthentifizierung \\
	\hline
	FA-103, FA-104
	& 100\%
	& Wichtige Sicherheitsfunktion, die wahrscheinlich stabil bleibt. \\
	\hline
\end{tabular}

\newpage

\subsection{Termine einsehen}
Dieses System ist in erster Linie dazu da um Termine zu verwalten. Bei der Betrachtung dieser Funktionalität spielen die verschiedenen Benutzergruppen bzw. die verschiedenen Sichten auf diese Funktionalitäten eine große Rolle. 

\vspace{12pt}

\subsubsection{Unangemeldeter Benutzer}
Es macht Sinn, das alle Termine von allen Personen eingesehen werden können, weil nicht davon auszugehen ist, dass jeder Student immer seine Anmeldedaten kennt. Zudem sind alle Termine zum jetzigen Zeitpunkt auch für alle zugänglich. So kann sicher gestellt werden, dass jeder Student, der eine Internetverbindung und ein Internet fähiges Gerät besitzt, heraus finden kann, ob er gleich einen Termin hat oder nicht. \\
Beim ITC werden für diese Kategorisierung die Jahrgänge und die Gruppen benutzt. Das nehmen wir als gegeben hin und haben vor es sehr ähnlich umzusetzen.

\vspace{12pt}

\begin{tabular} {|p{1,4cm}|p{10,7cm}|p{2,7cm}|}
	\hline
	ID & Anforderung & Kriterien \\
	\hline
	\printfreqnr
	& Wenn ein unangemeldeter Benutzer das System aufruft \textbf{SOLL} das System dem Benutzer die Möglichkeit geben alle Termine einzusehen. 
	& NFA-102, NFA-103 \\
	\hline
	\printfreqnr 
	& Wenn ein unangemeldeter Benutzer alle Termine einsieht \textbf{SOLL} das System dem Benutzer die Möglichkeit geben Jahrgänge auszuwählen.
	& -  \\
	\hline
	\printfreqnr 
	& Wenn ein unangemeldeter Benutzer alle Termine einsieht \textbf{SOLL} das System dem Benutzer die Möglichkeit geben die Gruppen \texttt{A-F} und \texttt{G-L} auszuwählen.
	& - \\
	\hline
	\printfreqnr
	& Wenn ein unangemeldeter Benutzer einen oder mehrere Jahrgänge ausgewählt hat, \textbf{MUSS} das System alle Termine anzeigen, die sich den ausgewählten Jahrgängen zuordnen lassen.
	& - \\
	\hline
	\printfreqnr
	& Wenn ein unangemeldeter Benutzer eine oder alle Gruppen ausgewählt hat \textbf{MUSS} das System alle Termine anzeigen, die sich den ausgewählten Gruppen zuordnen lassen.
	& - \\
	\hline
	\printfreqnr
	& Wenn ein unangemeldeter Benutzer sowohl Jahrgänge als auch Gruppen ausgewählt hat \textbf{MUSS} das System die Schnittmenge der beiden Termin-Menge anzeigen.
	& - \\
	\hline
	\printfreqnr
	& Wenn ein unangemeldeter Benutzer keine Gruppe ausgewählt hat \textbf{MUSS} das System diesen Filter ignorieren.
	& - \\
	\hline
	\printfreqnr
	& Wenn ein unangemeldeter Benutzer keine Jahrgänge ausgewählt hat \textbf{MUSS} das System diesen Filter ignorieren.
	& - \\
	\hline
\end{tabular}

\newpage

\subsubsection{Angemeldeter Student und Dozent}
Bei dem Einsehen von Terminen haben Studenten und Dozenten fast identische Anwendungsfälle und Anforderungen. Studenten und Dozenten rufen ihren Stundenplan auf um herauszufinden wann sie einen Termin haben. Es macht Sinn nur die Termine anzuzeigen, die für den angemeldeten Benutzer bestimmt sind. Dadurch muss der Benutzer weniger über seinen Stundenplan nachdenken, weniger filtern und kann wichtige Informationen direkt mit wenigen Blicken in Erfahrung bringen. 

\vspace{12pt}

\begin{tabular} {|p{1,4cm}|p{10,7cm}|p{2,7cm}|}
	\hline
	ID & Anforderung & Kriterien \\
	\hline
	\printfreqnr
	& Wenn ein Angemeldeter Student oder Dozent seine Termine einsehen möchte \textbf{MUSS} das System die Termine anzeigen, die sich dem Studenten oder Dozenten zuordnen lassen. 
	& - \\
	\hline
\end{tabular}

\newpage

\subsubsection{Angemeldeter Verwalter}
Aus unserer Sicht macht es Sinn, dass ein Verwalter alle Termine einsehen kann. Die Ansicht der Termine soll weitestgehend identisch mit der eines unangemeldeten Benutzers sein. Zusätzlich halten wir es für realistisch, dass ein Verwalter nach einem Studenten suchen muss. Dabei soll der Verwalter, eine Matrikelnummer oder einen Namen in ein Feld eingeben, dann einen konkreten Studenten aussuchen können. Danach sollen nur noch alle Termine des Ausgewählten Studenten angezeigt werden.

\vspace{12pt}

\begin{tabular} {|p{1,4cm}|p{10,7cm}|p{2,7cm}|}
	\hline
	ID & Anforderung & Kriterien \\
	\hline
	\printfreqnr
	& Wenn ein angemeldeter Verwalter alle Termine einsehen möchte \textbf{MUSS} das System alle Termine anzeigen. 
	& - \\
	\hline
	\printfreqnr
	& Wenn ein angemeldeter Verwalter alle Termine angezeigt bekommt \textbf{MUSS} dem Benutzer die Möglichkeit in einem Suchfeld eine Matrikelnummer oder einen Namen einzugeben. 
	& - \\
	\hline
	\printfreqnr
	& Wenn ein angemeldeter Verwalter in dem Suchfeld mindestens einen Buchstaben eingegeben hat, \textbf{MUSS} das System mögliche Studenten, bei denen die eingegebenen Zeichen entweder bei der Matrikelnummer oder bei dem Namen übereinstimmen, auflisten und zur Auswahl bereitstellen. 
	& - \\
	\hline
	\printfreqnr
	& Wenn ein angemeldeter Verwalter einen Studenten ausgewählt hat \textbf{MUSS} das System alle Termine des ausgewählten Studenten anzeigen. 
	& - \\
	\hline
	\printfreqnr
	& Wenn ein angemeldeter Verwalter einen Studenten ausgewählt hat \textbf{MUSS} das System dem Verwalter die Möglichkeit geben, den Studenten als Filter zu entfernen. 
	& - \\
	\hline
	\printfreqnr
	& Wenn ein angemeldeter Verwalter einen Studenten ausgewählt hat \textbf{DARF} das System \textbf{NICHT} dem Verwalter die Möglichkeit geben, einen weiteren Studenten auszuwählen. 
	& - \\
	\hline
\end{tabular}

\newpage

\subsubsection{Anzuzeigende Daten}
Für einen unangemeldeten Benutzer darf bei jedem Termin, das Datum, die Start- und Endzeit, der Dozent, die Bezeichnung des Moduls und falls vorhanden die Raumnummer angezeigt werden. 

\vspace{6pt}

Für einen angemeldeten Dozenten werden mit einer Ausnahme und einer Erweiterung die gleichen Daten für jeden Termin angezeigt. Das wären, das Datum, die Start- und Endzeit, die Bezeichnung des Moduls, falls vorhanden die Raumnummer und falls vorhanden der Link zu einer Onlineplattform. 

\vspace{6pt}

Für einen angemeldeten Studenten werden mit einer Erweiterung die gleichen Daten für jeden Termin angezeigt, wie für einen unangemeldeten Benutzer. Das wären, das Datum, die Start- und Endzeit, der Dozent, die Bezeichnung des Moduls, falls vorhanden die Raumnummer und falls vorhanden der Link zu einer Onlineplattform. 

\vspace{6pt}

Für einen angemeldeten Verwalter werden die gleichen Daten für jeden Termin angezeigt, wie bei einem angemeldeten Studenten. Das wären, das Datum, die Start- und Endzeit, der Dozent, die Bezeichnung des Moduls, falls vorhanden die Raumnummer und falls vorhanden der Link zu einer Onlineplattform. 

\vspace{12pt}

\begin{tabular} {|p{1,4cm}|p{10,7cm}|p{2,7cm}|}
	\hline
	ID & Anforderung & Kriterien \\
	\hline
	\printfreqnr
	& Wenn ein unangemeldeter Benutzer alle Termine angezeigt bekommt \textbf{MUSS} das System, das Datum, die Start- und Endzeit, der Dozent, die Bezeichnung des Moduls und falls vorhanden die Raumnummer anzeigen. 
	& - \\
	\hline
	\printfreqnr
	& Wenn ein angemeldeter Dozent alle Termine angezeigt bekommt \textbf{MUSS} das System, das Datum, die Start- und Endzeit, die Bezeichnung des Moduls, falls vorhanden die Raumnummer und falls vorhanden der Link zu einer Onlineplattform anzeigen. 
	& - \\
	\hline
	\printfreqnr
	& Wenn ein angemeldeter Student alle Termine angezeigt bekommt \textbf{MUSS} das System, das Datum, die Start- und Endzeit, der Dozent, die Bezeichnung des Moduls, falls vorhanden die Raumnummer und falls vorhanden der Link zu einer Onlineplattform anzeigen. 
	& - \\
	\hline
	\printfreqnr
	& Das System \textbf{DARF NICHT} den Link für Onlinevorlesungen für unangemeldete Benutzer anzeigen. 
	& - \\
	\hline
\end{tabular}

\newpage

\subsubsection{Mögliche Änderungen}
Hier sind die möglichen Änderungen für die Anforderungen in diesem Unterkapitel.

\vspace{6pt}

\begin{tabular} {|p{4cm}|p{2,1cm}|p{8,7cm}|}
	\hline
	Anforderungen & Stabilität & Begründung \\
	\hline
	FA-201
	& 90\%
	& Der Kalender soll auch Interessierten zur Verfügung stehen; es ist unwahrscheinlich, dass sich das ändert. \\
	\hline
	FA-202, FA-203, FA-204, FA-205, FA-206, FA-207, FA-208
	& 100\%
	& Sinnvolle Anforderung für bessere User Experience. \\
	\hline
	FA-209
	& 100\%
	& Grundfunktionalität, die stabil bleibt. \\
	\hline
	FA-210
	& 90\%
	& Grundanforderung, könnte aber auch später durch weitere Verwaltungsfunktionen erweitert werden. \\
	\hline
	FA-211, FA-212, FA-213, FA-214
	& 85\%
	& Praktisch stabil, könnte durch Änderungen in der Suchfunktionalität angepasst werden. \\
	\hline
	FA-215
	& 100\%
	& Klare logische Einschränkung, die stabil bleibt. \\
	\hline
	FA-216
	& 75\%
	& Klare logische Einschränkung, die stabil bleibt. \\
	\hline
	FA-217, FA-218
	& 80\%
	& Standardanforderung, könnte durch neue Anzeigekriterien angepasst werden. \\
	\hline
	FA-219
	& 100\%
	& Wichtige Sicherheitsanforderung, die stabil bleibt. \\
	\hline
\end{tabular}

\newpage

\subsection{Termine verwalten}
Unter Termine verwalten versteht man, die Bearbeitung, die Erstellung und das Löschen von Terminen. Dabei haben verschiedene Benutzergruppen verschiedene Anforderungen an die Verwaltung von Terminen.

\vspace{12pt}

\subsubsection{Unangemeldeter Benutzer und Studenten}
Es macht keinen Sinn, dass ein Student oder unangemeldeter Benutzer, Termine verwalten kann.

\vspace{12pt}

\begin{tabular} {|p{1,4cm}|p{10,7cm}|p{2,7cm}|}
	\hline
	ID & Anforderung & Kriterien \\
	\hline
	\printfreqnr
	& Das System \textbf{DARF NICHT} erlauben, dass Studenten oder unangemeldete Benutzer, Termine in irgendeiner Form verwalten. 
	& - \\
	\hline
	\printfreqnr
	& Wenn ein angemeldeter Student oder ein unangemeldeter Benutzer versucht einen Termine zu verwalten (bearbeiten, erstellen oder löschen), dann \textbf{MUSS} das System weiteres Vorgehen unterbinden und dem Benutzer mitteilen, dass er nicht autorisiert ist.
	& - \\ 
	\hline
\end{tabular}

%\vspace{12pt}
\newpage

\subsubsection{Angemeldeter Dozent}
Ein Dozent ist maßgeblich an der Terminfindung beteiligt. Bestimmte Richtlinien am ITC machen es jedoch etwas schwieriger passende Termine zu finden. Es ist z.B.: angedacht, dass ein Student mindestens zwei Tage pro Woche im Unternehmen arbeitet. Deshalb muss ein Verwalter, der solches Wissen besitzt auch an der Terminfindung beteiligt sein. Wir haben uns überlegt, dass es Sinn machen würde, wenn ein Dozent, Toolgestützt Termine als eine Art Vorschlag einreichen kann. Ein Vorschlag wäre dann eine Summe von Terminen. Dieser Vorschlag muss dann von einem Verwalter untersucht werden. In der Ansicht um Termine anzulegen müssen die bereits belegten Zeiträume eindeutig identifizierbar sein, damit es keine Überlappungen gibt und der Dozent Termine in noch freien Zeiträumen wählen kann. Unterschieden wird zwischen der Änderung der Daten (Start- und Endzeit, Modul Bezeichnung, Raumnummer etc.) und dem Erzeugen, Löschen und Änderungen an dem Datum. Um Termine zu erzeugen, löschen oder um das Datum zu ändern wird ein beschriebener Vorschlag benutzt. Um kleinere Daten, wie die Zeiten etc. zu Ändern wird so ein Vorschlag nicht benötigt. Dadurch soll es dem Dozenten möglich sein, schnell und mobil Termindaten zu ändern, ohne dabei ihm unbekannte Richtlinien zu verletzen.

\vspace{12pt}

\begin{tabular} {|p{1,4cm}|p{10,7cm}|p{2,7cm}|}
	\hline
	ID & Anforderung & Kriterien \\
	\hline
	\printfreqnr
	& Wenn ein angemeldeter Dozent Termine anlegen möchte \textbf{MUSS} das System einen neuen Vorschlag erstellen und diesen von dem Dozenten bearbeiten lassen.
	& - \\
	\hline
	\printfreqnr
	& Wenn ein angemeldeter Dozent einen Vorschlag bearbeitet, \textbf{MUSS} das System den Benutzer dazu auffordern \textbf{einen} Jahrgang auszuwählen. 
	& - \\
	\hline
	\printfreqnr
	& Wenn ein angemeldeter Dozent einen Vorschlag bearbeitet und ein Jahrgang ausgewählt wurde, \textbf{MUSS} das System dem Benutzer die Möglichkeit geben einen neuen Termin anzulegen.  
	& - \\
	\hline
	\printfreqnr
	& Wenn ein angemeldeter Dozent einen neuen Termin anlegt, \textbf{MUSS} das System den Benutzer fragen, für welche Veranstaltung der Termin gedacht ist.
	& - \\
	\hline
	\printfreqnr
	& Wenn ein angemeldeter Dozent einen neuen Termin anlegt und die Veranstaltung ausgewählt hat, \textbf{MUSS} das System den Benutzer die Möglichkeit geben, die restlichen Daten (Start- und Endzeit, Raumnummer, Link für die Vorlesung) einzutragen. 
	& - \\
	\hline
\end{tabular}
	
\begin{tabular} {|p{1,4cm}|p{10,7cm}|p{2,7cm}|}
	\hline
	ID & Anforderung & Kriterien \\
	\hline
	\printfreqnr
	& Wenn ein angemeldeter Dozent alle benötigten Daten eingegeben hat, \textbf{MUSS} das System verifizieren, dass der neue Termin in einem freien Zeitraum liegt.
	& - \\
	\hline
	\printfreqnr
	& Wenn das System verifiziert hat, dass der Termin in einem freien Zeitraum liegt, \textbf{MUSS} das System versuchen den Termin in dem Vorschlag zu speichern.
	& - \\
	\hline
	\printfreqnr
	& Das System feststellt, dass der Termin in einem belegten Zeitraum liegt, \textbf{MUSS} das System weiteres Vorgehen unterbinden und den Dozenten darauf hinweisen, dass der Termin in diesem Zeitraum nicht angelegt werden kann.
	& - \\
	\hline
	\printfreqnr
	& Wenn ein angemeldeter Dozent mindestens einen Termin angelegt hat, \textbf{MUSS} das System dem Dozenten die Möglichkeit geben den Vorschlag freizugeben, damit ein Verwalter diesen einsehen kann.
	& - \\
	\hline
	\printfreqnr
	& Ein Dozent \textbf{MUSS} einen seiner angelegten Termine im Nachhinein bezüglich der Raumnummer oder des Links zur Online-Veranstaltung selbst ändern können.
	& - \\
	\hline
\end{tabular}


%\newpage
\vspace{12px}

Um diesen Prozess für den Dozenten so angenehm wie möglich zu machen, soll es Tools geben, die Ihm das Leben einfacher machen sollen. Zuerst soll es eine KI geben, die Zeiträume für Termine vorschlägt. Dabei ist das Ziel, dass der Dozent Termine auswählen kann, zu denen die Studenten z.B.: am aktivsten sind. 
%Das zweite Tool ist eine Art Wochenplaner, bei dem der Dozent eine für das Semester repräsentative Woche mit Terminen einträgt und das Tool generiert dann einen einreichbaren Vorschlag. Dadurch soll dem Dozenten Arbeit abgenommen werden.

\vspace{12pt}

\begin{tabular} {|p{1,4cm}|p{10,7cm}|p{2,7cm}|}
	\hline
	ID & Anforderung & Kriterien \\
	\hline
	\printfreqnr
	& Wenn ein angemeldeter Dozent einen Vorschlag bearbeitet, \textbf{MUSS} das System dem Benutzer auf KI-basierende Termine Vorschlagen. 
	& - \\
	\hline
	\printfreqnr
	& Wenn ein angemeldeter Dozent einen auf KI-basierenden Termin auswählt \textbf{MUSS} das System die gleichen Daten, wie bei dem Normalen anlegen fragen. Die einzige Ausnahme ist der Zeitraum. Dieser soll vor befüllt aber bearbeitbar sein. 
	& - \\
	\hline
\end{tabular}


\newpage

\subsubsection{Angemeldeter Verwalter}
Der Verwalter hat deutlich mehr recht was die Verwaltung von Terminen angeht, weil ein Verwalter dazu in der Lage sein muss den gesamten Stundenplan zu reparieren. 

\vspace{12pt}

\begin{tabular} {|p{1,4cm}|p{10,7cm}|p{2,7cm}|}
	\hline
	ID & Anforderung & Kriterien \\
	\hline
	\printfreqnr
	& Wenn ein angemeldeter Dozent einen Vorschlag eingereicht hat, \textbf{MUSS} das System dem angemeldeten Verwalter den Vorschlag anzeigen und Ihm die Möglichkeit geben den Vorschlag anzuwenden. 
	& - \\
	\hline
	\printfreqnr
	& Wenn der angemeldeter Verwalter den Vorschlag anwendet, \textbf{MUSS} das System versuchen die enthaltenen Termine zu persistieren.
	& - \\
	\hline
	\printfreqnr
	& Wenn der angemeldeter Verwalter den Vorschlag anwendet, \textbf{MUSS} das System sicherstellen, dass es keine Überlappungen mit anderen Terminen gibt.
	& - \\
	\hline
	\printfreqnr
	& Wenn der angemeldeter Verwalter den Vorschlag ablehnt, \textbf{MUSS} das System den Verwalter dazu auffordern einen Grund anzugeben und den betroffenen Dozenten benachrichtigen.
	& - \\
	\hline
	\printfreqnr
	& Wenn der angemeldeter Verwalter einen neuen Termin anlegt, \textbf{MUSS} das System sicher stellen, dass es keine Überlappungen mit anderen Terminen bei beteiligten Studenten gibt.
	& - \\
	\hline
	\printfreqnr
	& Wenn es keine Überlappungen geben sollte, \textbf{MUSS} das System versuchen den Termin anzulegen.
	& - \\
	\hline
	\printfreqnr
	& Wenn es Überlappungen geben sollte, \textbf{MUSS} das System den Verwalter dazu auffordern einen anderen Zeitraum zu wählen.
	& - \\
	\hline
	\printfreqnr
	& Ein Verwalter \textbf{MUSS} ebenfalls die Möglichkeit haben Termine über einen Vorschlag mit dem Dozenten abzustimmen.
	& - \\
	\hline
	\printfreqnr
	& Wenn ein Verwalter versucht einen Termin zu verändern oder zu löschen \textbf{MUSS} das System versuchen den Termin zu verändern oder zu löschen.
	& - \\
	\hline
	\printfreqnr
	& Wenn ein Verwalter versucht einen Termin zu verändern \textbf{MUSS} das System sicher stellen, dass es keine Überlappungen mit anderen Terminen gibt.
	& - \\
	\hline
\end{tabular}

\newpage

\subsubsection{Mögliche Änderungen}
Hier sind die möglichen Änderungen für die Anforderungen in diesem Unterkapitel.

\vspace{6pt}

\begin{tabular} {|p{4cm}|p{2,1cm}|p{8,7cm}|}
	\hline
	Anforderungen & Stabilität & Begründung \\
	\hline
	FA-301, FA-302
	& 100\%
	& Klare Sicherheitsanforderung, die stabil bleibt. \\
	\hline
	FA-303
	& 40\%
	& Diese Anforderung könnte durch automatisierte Vorschläge und KI-basierte Unterstützung ergänzt werden. \\
	\hline
	FA-304, FA-305, FA-306
	& 70\%
	& Hinsichtlich der Kernfunktion stabil, User Experience könnte jedoch verbessert werden, z.B. durch eine intelligente Standardwertauswahl. \\
	\hline
	FA-307
	& 30\%
	& Hinsichtlich der Kernfunktion stabil, User Experience könnte jedoch verbessert werden, z.B. durch eine intelligente Standardwertauswahl. \\
	\hline
	FA-308, FA-309
	& 100\%
	& Sinnvolle Anforderung, die stabil bleibt. \\
	\hline
	FA-310
	& 60\%
	& Sinnvolle Anforderung, die stabil bleibt. Gegebenenfalls könnte hier bereits ein Ersatztermin vorgeschlagen und der Grund der kollidierende Termin angezeigt werden. Zudem könnte es sinnvoll sein, aus dieser Sicht auswählen zu können, welcher Termin gelten soll, der bestehende oder der neue. \\
	\hline
	FA-311
	& 70\%
	& Möglicherweise könnte hier auch bereits eine Nachricht an betroffene Studenten und die Verwaltung rausgehen. \\
	\hline
	FA-312
	& 10\%
	& Es kann sich auch herausstellen, dass eine Änderung der Termine durch den Dozenten insgesamt nicht erwünscht sind, und stattdessen nur Änderungsvorschläge gemacht werden, welche durchd ie Verwaltung genehmigt werden müssen. \\
	\hline
	FA-313
	& 100\%
	& Sinnvolle Kernanforderung, die stabil bleibt. \\
	\hline
	FA-314
	& 30\%
	& Eventuell wird die KI später auch eine intelligente Vorauswahl aller Werte vornehmen, und bloß noch Korrekturen abfragen. \\
	\hline
\end{tabular}

\begin{tabular} {|p{4cm}|p{2,1cm}|p{8,7cm}|}
	\hline
	Anforderungen & Stabilität & Begründung \\
	\hline
	FA-315
	& 85\%
	& Praktisch stabil, könnte jedoch durch Änderungen in der Genehmigungslogik angepasst werden. \\
	\hline
	FA-316, FA-317, FA-318, FA-319, FA-320, FA-321, FA-324
	& 85\%
	& Praktisch stabil, könnte jedoch durch Änderungen in der Logik zur Vermeidung von Terminkollisionen angepasst werden. \\
	\hline
	FA-322, FA-323
	& 100\%
	& Praktisch stabile Anforderung. \\
	\hline
\end{tabular}

\newpage

\subsection{Termine verschieben}
Falls ein Termin nachträglich verschoben werden muss, muss ein Prozess vorhanden sein, der diese Änderung in den Datenbestand einpflegt und dabei Rechte und die Plausibilität beachtet.

\subsubsection{Unangemeldeter Benutzer und Student}
Ein angemeldeter Student oder ein unangemeldeter Benutzer darf unter keinen Umständen Termine verschieben.

\vspace{12pt}

\begin{tabular} {|p{1,4cm}|p{10,7cm}|p{2,7cm}|}
	\hline
	ID & Anforderung & Kriterien \\
	\hline
	\printfreqnr
	& Das System \textbf{DARF NICHT} erlauben, dass Studenten oder unangemeldete Benutzer, Termine in irgendeiner Form verschieben. 
	& - \\
	\hline
	\printfreqnr
	& Wenn ein angemeldeter Student oder ein unangemeldeter Benutzer versucht einen Termine zu verschieben, dann \textbf{MUSS} das System weiteres Vorgehen unterbinden und dem Benutzer mitteilen, dass er nicht autorisiert ist.
	& - \\ 
	\hline
\end{tabular}

\vspace{12pt}

\subsubsection{Angemeldeter Dozent}
Ein Dozent darf Termine in ihrer Uhrzeit selbst verschieben, soll aber für Änderungen am Datum eines Termins immer die Verwaltung anfragen müssen.

\vspace{12pt}

\begin{tabular} {|p{1,4cm}|p{10,7cm}|p{2,7cm}|}
	\hline
	ID & Anforderung & Kriterien \\
	\hline
	\printfreqnr
	& Ein Dozent \textbf{MUSS} die Uhrzeit eines Termins ändern können.
	& - \\
	\hline
	\printfreqnr
	& Ein Dozent \textbf{DARF NICHT} selbst das Datum eines Termins ändern.
	& - \\
	\hline
	\printfreqnr
	& Ein Dozent \textbf{MUSS} eine Anfrage zum Ändern des Datums eines Termins an die Verwaltung stellen können.
	& - \\
	\hline
	\printfreqnr
	& Ein Dozent \textbf{SOLL} beim Formulieren einer Verschieben-Anfrage für einen Termin, für den noch eine Anfrage aussteht, einen entsprechenden Hinweis darauf erhalten.
	& - \\
	\hline
	\printfreqnr
	& Ein Dozent \textbf{SOLL} eine Vorschau von ausstehenden Anfragen in seiner Terminansicht haben. Dabei sollte der ursprüngliche Termin als vorläufig nicht mehr aktuell gekennzeichnet werden.
	& - \\
	\hline
\end{tabular}

\newpage

\subsubsection{Angemeldeter Verwalter}
Ein Verwalter darf Termine selbst beliebig verschieben und muss die von Dozenten gestellten Anfragen bearbeiten können. Ebenfalls muss der Verwalter die von Dozenten selbst durchgeführten Verschiebungen kontrollieren können.

\vspace{12pt}

\begin{tabular} {|p{1,4cm}|p{10,7cm}|p{2,7cm}|}
	\hline
	ID & Anforderung & Kriterien \\
	\hline
	\printfreqnr
	& Ein Verwalter \textbf{MUSS} einen Termin in seiner Uhrzeit und seinem Datum verschieben können.
	& - \\
	\hline
	\printfreqnr
	& Während des Verschiebens-Prozesses \textbf{SOLL} klar erkennbar sein, wenn es dabei zu Überschneidungen mit anderen Terminen kommt.
	& - \\
	\hline
	\printfreqnr
	& Ein Verwalter \textbf{MUSS} die gestellten, noch nicht angenommenen, Datumsänderungsanfragen von Dozenten einsehen können.
	& - \\
	\hline
	\printfreqnr
	& Ein Verwalter \textbf{MUSS} die Verschiebe-Anfragen von Dozenten annehmen oder ablehnen können, je nachdem, ob sich diese mit anderen Terminen überschneiden oder durch den neuen Termin Richtlinien verletzt werden würden.
	& - \\
	\hline
	\printfreqnr
	& Ein Verwalter \textbf{SOLL} einfach in der Übersicht erkennen können, ob die Uhrzeit eines Termins eigenmächtig von einem Dozenten verschoben wurde.
	& - \\
	\hline
	\printfreqnr
	& Ein Verwalter \textbf{MUSS} Details zu den durchgeführten Uhrzeitänderungen von Dozenten einsehen können und diese rückgängig machen, falls diese zu Konflikten führen.
	& - \\
	\hline
\end{tabular}

\newpage

\subsubsection{Mögliche Änderungen}
Hier sind die möglichen Änderungen für die Anforderungen in diesem Unterkapitel.

\vspace{6pt}

\begin{tabular} {|p{4cm}|p{2,1cm}|p{8,7cm}|}
	\hline
	Anforderungen & Stabilität & Begründung \\
	\hline
	FA-401, FA-402, FA-404, FA-405, FA-406, FA-408, FA-410, FA-411
	& 100\%
	& Diese Anforderung ist Ausdruck der grundlegenden Designentscheidung, Termine zentral über die Verwalter betreuen zu lassen. \\
	\hline
	FA-403, FA-412, FA-413
	& 80\%
	& Praktisch stabile Anforderung. \\
	\hline
	FA-407
	& 50\%
	& Eventuell wird der Änderungsantrag nur vermerkt, da ein "nicht mehr aktuell"-Vermerk zu Missverständnissen führen könnte. \\
	\hline
	FA-409
	& 80\%
	& Die Forderung bleibt in dieser Form bestehen, damit es nicht zu Konkurrenzen kommt. Eventuell muss noch deutlich gemacht werden, dass manche Termine ja durchaus gleichzeitig stattfinden können, z.B. alternative Wahlplfichtfächer. \\
	\hline
\end{tabular}

\newpage

\subsection{Anwesenheitsliste verwalten}

\subsubsection{Studenten und unangemeldete Benutzer}
Angemeldete Studenten und unangemeldete Benutzer dürfen unter keinen Umständen die Anwesenheitsliste in irgendeiner Form verwalten. Es darf nicht möglich sein, dass ein Student seine eigenen Fehlzeiten bearbeiten kann.

\vspace{12pt}

\begin{tabular} {|p{1,4cm}|p{10,7cm}|p{2,7cm}|}
	\hline
	ID & Anforderung & Kriterien \\
	\hline
	\printfreqnr
	& Das System \textbf{DARF NICHT} erlauben, dass Studenten oder unangemeldete Benutzer, Anwesenheitslisten in irgendeiner Form verwalten oder einsehen. 
	& - \\
	\hline
	\printfreqnr
	& Wenn ein angemeldeter Student oder ein unangemeldeter Benutzer versucht eine Anwesenheitsliste zu verwalten oder einzusehen, dann \textbf{MUSS} das System weiteres Vorgehen unterbinden und dem Benutzer mitteilen, dass er nicht autorisiert ist.
	& - \\ 
	\hline
\end{tabular}

\newpage

\subsubsection{Angemeldeter Dozent}
Das Eintragen der Anwesenheit eines Studenten durch den Dozenten stellt eine Kernfunktionalität der Software da. Der Dozent sollte im Zeitraum der Vorlesung in der Lage sein, auf verschiedene Arten mit der Anwesenheitsliste interagieren zu können. Pro Student soll es ebenfalls möglich sein, Kommentare zu hinterlegen, falls es zu besonderen Umständen kommt, die nicht akkurat mit der Anwesenheitsliste abgebildet werden können. Die Bearbeitung einer Anwesenheitsliste soll im nur dem Zeitraum der entsprechenden Termins möglich seien, da für nachträgliche Bearbeitung nur ein Verwalter berechtigt seien soll.
\vspace{12pt}

\begin{tabular} {|p{1,4cm}|p{10,7cm}|p{2,7cm}|}
	\hline
	ID & Anforderung & Kriterien \\
	\hline
	\printfreqnr
	& Ein Dozent \textbf{SOLL} auf der Startseite einen anklickbaren Reiter haben, welcher die Anwesenheitsliste seines aktuellen Termins öffnet.
	& - \\
	\hline
	\printfreqnr
	& Das System \textbf{SOLL} einen Termin als aktuell erkennen, wenn die Systemzeit im Zeitrahmen des Termins liegt, inklusive mehrere Minuten Puffer vor dem Terminanfang und nach dem Terminende.
	& - \\
	\hline
	\printfreqnr
	& Ein Dozent \textbf{SOLL} auf der Anwesenheitsliste einen Reiter für vergangene Termine haben, auf welcher er die Anwesenheitsliste dieser einsehen kann.
	& - \\
	\hline
	\printfreqnr
	& Ein Dozent \textbf{SOLL} nur im Zeitrahmen eines Termins in der Lage sein, die jeweilige Anwesenheitsliste zu bearbeiten.
	& - \\
	\hline
	\printfreqnr
	& Ein Dozent \textbf{MUSS} in der Lage sein, die Anwesenheit eines Studenten einzutragen.
	& - \\
	\hline
	\printfreqnr
	& Ein Dozent \textbf{MUSS} in der Lage sein, den Eintrag der Anwesenheit eines Studenten im Laufe des Termins bearbeiten zu können. 
	& - \\
	\hline
	\printfreqnr
	& Ein Dozent \textbf{MUSS} in der Lage sein, den aktuellen Stand der Anwesenheitsliste einsehen zu können.
	& - \\
	\hline
	\printfreqnr
	& Ein Dozent \textbf{MUSS} in der Lage sein, pro Termin für jeden Studenten Kommentare hinterlegen zu können.
	& - \\
	\hline
\end{tabular}

\newpage

\subsubsection{Angemeldeter Verwalter}
Wenn ein Student fälschlicherweise eingetragen/nicht eingetragen wurde, muss ein Verwalter in der Lage sein, dies nachträglich zu bearbeiten. Ebenfalls soll der Verwalter die hinterlegten Kommentare einsehen zu können.
\vspace{12pt}

\begin{tabular} {|p{1,4cm}|p{10,7cm}|p{2,7cm}|}
	\hline
	ID & Anforderung & Kriterien \\
	\hline
	\printfreqnr
	& Ein Verwalter \textbf{SOLL} auf der Startseite einen anklickbaren Reiter haben, welche eine Übersicht aller Termine öffnet.
	& - \\
	\hline
	\printfreqnr
	& Ein Verwalter \textbf{MUSS} in der Lage sein, eine Anwesenheitsliste nachträglich bearbeiten zu können
	& - \\
	\hline
	\printfreqnr
	& Ein Verwalter \textbf{MUSS} in der Lage sein, die hinterlegten Kommentare einer Anwesenheitsliste einsehen zu können.
	& - \\
	\hline
\end{tabular}

\newpage
%\vspace{12px}

\subsubsection{Mögliche Änderungen}
Hier sind die möglichen Änderungen für die Anforderungen in diesem Unterkapitel.

\vspace{6pt}

\begin{tabular} {|p{4cm}|p{2,1cm}|p{8,7cm}|}
	\hline
	Anforderungen & Stabilität & Begründung \\
	\hline
	FA-501, FA-502
	& 100\%
	& Studenten sollen sich nicht selbst eintragen, die Anwesenheit soll nur vom Dozenten bestätigt werden. \\
	\hline
	FA-503, FA-505, FA-511
	& 90\%
	& Praktisch stabil, könnte jedoch durch Änderungen in der Benutzeroberfläche angepasst werden. \\
	\hline
	FA-504
	& 90\%
	& Praktisch stabil, könnte jedoch durch Änderungen in der Systemlogik angepasst werden. \\
	\hline
	FA-506
	& 100\%
	& Jede Anwesenheitsliste gehört notwendigerweise zu einem Termin. \\
	\hline
	FA-507, FA-508, FA-509, FA-510
	& 100\%
	& Grundfunktionalität, die stabil bleibt. \\
	\hline
	FA-512, FA-513
	& 100\%
	& Grundfunktionalität, die stabil bleibt. Die Bearbeitung muss möglich sein, da Entschuldigungen ja ggf. nachgereicht werden. \\
	\hline
\end{tabular}

\newpage

\subsection{Fehlzeiten einsehen}
Eine Fehlzeit ist ein Termin, bei dem der Student hätte da sein müssen. Die Fehlzeiten werden hauptsächlich von dem System selbst erzeugt. Dafür werden neue Anwesenheitslisten von Terminen Ausgewertet. Diese Auswertung findet jede Nacht statt. Ausgewertet werden nur neue oder bearbeitete Anwesenheitslisten. Bei neuen Anwesenheitslisten werden auch neue Fehlzeiten erzeugt. Bei Aktualisierungen wird lediglich die Konsistenz überprüft. Bei erzeugten Fehlzeiten gibt es einen Status. Es gibt \texttt{Unentschuldigt} oder \texttt{Entschuldigt}. Alle neu hinzugefügten Fehlzeiten sind zuerst Unentschuldigt. Wenn eine Fehlzeit existiert war der Student nicht anwesend. Über die Anwesenheitsliste kann der Status \textbf{NICHT} aktiv gesetzt werden. Sollte eine Anwesenheitsliste bearbeitet worden sein, so wird diese nachts erneut ausgewertet. Dabei werden lediglich die ausgewerteten Fehlzeiten mit den existierenden überprüft. Sollten hierbei Inkonsistenzen auffallen, so werden alle Verwalter über die Inkonsistenz benachrichtigt. Zudem findet die Prüfung auf Inkonsistenzen jeden Montag Morgen statt. Hierbei werden aber alle Anwesenheitslisten auf Inkonsistenzen geprüft. Diese vollständige Prüfung findet Montag statt, weil die Fehlzeiten spätestens Samstag Abend erzeugt werden. Zudem wird davon ausgegangen, dass das System am Montag Morgen wenig Ausgelastet ist. Bei der Prüfung wird geprüft, dass jede Fehlzeit aus allen Anwesenheitslisten erzeugt wurden und dass jede erzeugte Fehlzeit einer Anwesenheitsliste zugeordnet werden kann. Falls hierbei Inkonsistenzen gefunden werden, werden alle Verwalter benachrichtigt. 

\vspace{12pt}

\subsubsection{Unangemeldete Benutzer}
Ein unangemeldeter Benutzer darf unter keinen Umständen die Fehlzeiten von Benutzern einsehen.

\vspace{12pt}

\begin{tabular} {|p{1,4cm}|p{10,7cm}|p{2,7cm}|}
	\hline
	ID & Anforderung & Kriterien \\
	\hline
	\printfreqnr
	& Das System \textbf{DARF NICHT} erlauben, dass unangemeldete Benutzer, Fehlzeiten in irgendeiner Form einsehen. 
	& - \\
	\hline
	\printfreqnr
	& Wenn ein unangemeldeter Benutzer versucht Fehlzeiten einzusehen, \textbf{MUSS} das System weiteres Vorgehen unterbinden und dem Benutzer mitteilen, dass er nicht autorisiert ist.
	& - \\ 
	\hline
\end{tabular}

%\vspace{12pt}
\newpage

\subsubsection{Angemeldeter Student}
Wegen Datenschutz ist es wichtig, dass ein Student nur seine eigenen Fehlzeiten einsehen kann. Zugriff auf die Fehlzeiten anderer Studenten muss dementsprechend vom System unterbunden werden. 

\vspace{12pt}

\begin{tabular} {|p{1,4cm}|p{10,7cm}|p{2,7cm}|}
	\hline
	ID & Anforderung & Kriterien \\
	\hline
	\printfreqnr
	& Ein Student \textbf{MUSS} seine Fehlzeiten einsehen können. 
	& - \\
	\hline
	\printfreqnr
	& Wenn ein Student seine Fehlzeiten einsieht, \textbf{MUSS} das System die unentschuldigten Fehlzeiten mit der Farbe Rot markieren, damit auf dem ersten Blick erkennbar ist, welche Fehlzeiten noch entschuldigt werden müssen. 
	& - \\
	\hline
	\printfreqnr
	& Wenn ein Student seine Fehlzeiten einsieht, \textbf{SOLL} das System die entschuldigten Fehlzeiten mit der Farbe Grün markieren, damit ein Kontrast zu den unentschuldigten Terminen zusehen ist. 
	& - \\
	\hline
	\printfreqnr
	& Die angezeigten Fehlzeiten \textbf{MÜSSEN} notwendige Informationen über den Termin enthalten. (Siehe anzuzeigende Daten von den Terminen)
	& - \\ 
	\hline
	\printfreqnr
	& Wenn ein Student seine Termine einsieht, \textbf{SOLL} das System anzeigen, wenn Fehlzeiten für Termine vorliegen.
	& - \\ 
	\hline
\end{tabular}

\vspace{12pt}

\subsubsection{Angemeldeter Dozent}
Die Fehlzeiten beziehen Sich auf einzelne Studenten. Die Anwesenheitsliste wird als etwas völlig anderes gesehen. Es macht keinen Sinn, dass ein Dozent die Fehlzeiten von einzelnen Studenten einsehen kann. Der Zugriff muss also vom System unterbunden werden.

\vspace{12pt}

\begin{tabular} {|p{1,4cm}|p{10,7cm}|p{2,7cm}|}
	\hline
	ID & Anforderung & Kriterien \\
	\hline
	\printfreqnr
	& Ein Dozent \textbf{DARF NICHT} Fehlzeiten einsehen. Hier ist die Anwesenheitsliste eine Ausnahme. 
	& - \\
	\hline
\end{tabular}

%\vspace{12pt}
\newpage

\subsubsection{Angemeldeter Verwalter}
Ein Verwalter ist die höchste Instanz in unserem System und muss deshalb die Möglichkeit haben, Termine zu erzeugen, zu löschen und zu bearbeiten. Hierbei können besagte Inkonsistenzen auftreten.

\vspace{12pt}

\begin{tabular} {|p{1,4cm}|p{10,7cm}|p{2,7cm}|}
	\hline
	ID & Anforderung & Kriterien \\
	\hline
	\printfreqnr
	& Ein Verwalter \textbf{MUSS} alle Fehlzeiten einsehen können. 
	& - \\
	\hline
	\printfreqnr
	& Ein Verwalter \textbf{MUSS} alle Fehlzeiten zu einem spezifizierten Studenten in Erfahrung bringen können.
	& - \\ 
	\hline
	\printfreqnr
	& Wenn ein Verwalter anfängt die Matrikelnummer oder den Namen des Studenten einzugeben, \textbf{MUSS} das System passende Studenten vorschlagen.
	& - \\ 
	\hline
	\printfreqnr
	& Wenn ein Verwalter einen Studenten ausgewählt hat, \textbf{MUSS} das System alle Fehlzeiten des ausgewählten Studenten anzeigen.
	& - \\ 
	\hline
	\printfreqnr
	& Ein Verwalter, \textbf{MUSS} Studenten entschuldigen können.
	& - \\ 
	\hline
	\printfreqnr
	& Wenn ein Verwalter einen Studenten entschuldigt, \textbf{MUSS} das System die Fehlzeit als entschuldigt markieren.
	& - \\ 
	\hline
	\printfreqnr
	& Wenn eine Fehlzeit als entschuldigt markiert wird, \textbf{MUSS} das System den Studenten darüber benachrichtigen.
	& - \\ 
	\hline
	\printfreqnr
	& Wenn ein Verwalter Fehlzeit einsieht, \textbf{MUSS} das System die Fehlzeiten in einer Tabelle auflisten. Dabei existiert für jede Fehlzeit eine Zeile.
	& - \\ 
	\hline
	\printfreqnr
	& Wenn ein Verwalter Fehlzeiten einsieht, \textbf{MUSS} das System alle Termin relevanten Daten (Siehe Termine Anzeigen), den vollständigen Namen des Studenten, die Matrikelnummer des Studenten, den vollständigen Namen des Dozenten und den Status der Fehlzeit anzeigen.
	& - \\ 
	\hline
	\printfreqnr
	& Ein Verwalter \textbf{MUSS} eine neue Fehlzeit anlegen können. 
	& - \\ 
	\hline
	\printfreqnr
	& Wenn ein Verwalter eine neue Fehlzeit anlegt, \textbf{MUSS} der Verwalter einen Termin und einen Studenten auswählen können. 
	& - \\ 
	\hline
	\printfreqnr
	& Ein Verwalter \textbf{MUSS} eine Fehlzeit aus der Übersicht auswählen und löschen können. 
	& - \\ 
	\hline
	\printfreqnr
	& Wenn ein Verwalter eine Fehlzeit löscht oder hinzufügt, \textbf{MUSS} das System den Verwalter darauf hinweisen, dass ggf. noch die Anwesenheitsliste angepasst werden muss. Dieser Hinweis wird grundsätzlich immer angezeigt. Das System überprüft \textbf{NICHT} ob die Anwesenheitsliste im Vorhinein bearbeitet wurde.
	& - \\ 
	\hline
\end{tabular}

%\vspace{12pt}
\newpage

\subsubsection{Überschneidung mit Anwesenheitsliste}
Die Anwesenheitslisten sollten mit den Erzeugten Fehlzeiten übereinstimmen. Sowohl nachts um 23 Uhr als auch Montag Morgens um 3 Uhr ist die Serverlast gering. Deshalb sollte die Auswertung der Anwesenheitslisten und die Konsistenzprüfung zu solchen Zeiten stattfinden. Es ist wichtig, dass die Auswertung (Die Erzeugung der Termine) Täglich passiert, damit die Studenten rechtzeitig über die Fehlzeit benachrichtigt werden.

\vspace{12pt}

\begin{tabular} {|p{1,4cm}|p{10,7cm}|p{2,7cm}|}
	\hline
	ID & Anforderung & Kriterien \\
	\hline
	\printfreqnr
	& Das System \textbf{MUSS} jede Nacht um 23:00 Uhr alle neuen und bearbeiteten Anwesenheitslisten auswerten. 
	& - \\
	\hline
	\printfreqnr
	& Wenn es neue Anwesenheitslisten gibt, \textbf{MUSS} das System die nicht anwesenden Studenten identifizieren und Fehlzeiten anlegen. 
	& - \\ 
	\hline
	\printfreqnr
	& Wenn eine neue Fehlzeit angelegt wird, \textbf{MUSS} das System den Studenten über die Fehlzeit benachrichtigen. 
	& - \\ 
	\hline
	\printfreqnr
	& Wenn Anwesenheitslisten bearbeitet wurden, \textbf{MUSS} das System prüfen, ob alle ausgewerteten Fehlzeiten existieren. 
	& - \\ 
	\hline
	\printfreqnr
	& Das System \textbf{MUSS} jeden Montag um 3:00 Uhr die Konsistenz zwischen den Anwesenheitslisten und den Fehlzeiten vollständig prüfen.
	& - \\ 
	\hline
	\printfreqnr
	& Das System \textbf{MUSS} überprüfen, ob jede Fehlzeit in einer Anwesenheitsliste zu finden ist.
	& - \\ 
	\hline
	\printfreqnr
	& Das System \textbf{MUSS} überprüfen, ob jede Fehlzeit aus einer Anwesenheitsliste existiert.
	& - \\ 
	\hline
	\printfreqnr
	& Wenn das System Inkonsistenzen gefunden hat, \textbf{MUSS} das System alle Verwalter benachrichtigen.
	& - \\ 
	\hline
	\printfreqnr
	& Der Benachrichtigung, \textbf{MUSS} die Anwesenheitsliste, der Student und der Termin entnehmbar sein.
	& - \\ 
	\hline
\end{tabular}

\newpage

\subsubsection{Mögliche Änderungen}
Hier sind die möglichen Änderungen für die Anforderungen in diesem Unterkapitel.

\vspace{6pt}

\begin{tabular} {|p{4cm}|p{2,1cm}|p{8,7cm}|}
	\hline
	Anforderungen & Stabilität & Begründung \\
	\hline
	FA-601, FA-608
	& 100\%
	& Diese Anforderung ist auch Ausfluss des Datenschutzes und bleibt stabil. \\
	\hline
	FA-602, FA-603, FA-606, FA-607, FA-609, FA-610, FA-612, FA-613, FA-614, FA-615, FA-618, FA-619, FA-620, FA-621, FA-623, FA-624, FA-625, FA-626, FA-627, FA-628, FA-630
	& 100\%
	& Grundfunktionalität, die stabil bleibt. \\
	\hline
	FA-604, FA-605
	& 80\%
	& Praktisch stabil, könnte jedoch durch Änderungen in der Benutzeroberfläche angepasst werden. \\
	\hline
	FA-611
	& 100\%
	& UX-Funktionalität, die stabil bleibt. \\
	\hline
	FA-616
	& 60\%
	& UI-Funktionalität, die sich aufgrund von Nutzerrückmeldungen ändern kann. \\
	\hline
	FA-617
	& 70\%
	& UI-Funktionalität, die sich aufgrund von Nutzerrückmeldungen ändern kann. \\
	\hline
	FA-622
	& 40\%
	& Eventuell wird die Auswertung auch in Echtzeit vorgenommen. \\
	\hline
	FA-629
	& 80\%
	& Grundfunktionalität, die stabil bleibt. Es müssen eventuell nicht alle Verwalter benachrichtigt werden. \\
	\hline
\end{tabular}

\newpage

\subsection{Benutzer verwalten}
Die Anmeldedaten und Berechtigungen von Studenten, Dozenten und Verwalter, zusammengefasst die Benutzer des Systems, müssen verwaltet werden können.

Nur angemeldete Benutzer in der Rolle Verwalter sollen dazu in der Lage sein.

\vspace{12pt}

\subsubsection{Unangemeldete Benutzer, Studenten und Dozenten}
Normale Benutzer dürfen keine anderen Benutzer hinzufügen oder solche wie auch sich selbst bearbeiten oder löschen.

\vspace{12pt}


\begin{tabular} {|p{1,4cm}|p{10,7cm}|p{2,7cm}|}
	\hline
	ID & Anforderung & Kriterien \\
	\hline
	\printfreqnr
	& Das System \textbf{DARF NICHT} erlauben, dass Benutzer, die keine Verwalter sind, andere Benutzer oder sich selbst verwalten. 
	& - \\
	\hline
	\printfreqnr
	& Wenn ein nicht-autorisierter Benutzer versucht, Benutzer zu bearbeiten, \textbf{MUSS} das System eine Fehlermeldung zurückgeben.
	& - \\ 
	\hline
	\printfreqnr
	& Die Fehlermeldung \textbf{DARF} dem nicht autorisierten Benutzer \textbf{NICHT} mitteilen, dass der aktuelle Endpunkt zum Verwalten von Benutzern genutzt wird.
	& - \\ 
	\hline
\end{tabular}

\vspace{12pt}

\subsubsection{Angemeldeter Verwalter}
Es ist die Aufgabe der Verwalter, die Benutzerkonten für das System zu pflegen:

Sie müssen in der Lage sein, Benutzerkonten anzulegen, zu bearbeiten und wieder zu entfernen.

\vspace{12pt}

\begin{tabular} {|p{1,4cm}|p{10,7cm}|p{2,7cm}|}
	\hline
	ID & Anforderung & Kriterien \\
	\hline
	\printfreqnr
	& Das System \textbf{SOLL} den Verwaltern eine Liste der im System vorhandenen Benutzer anzeigen.
	& - \\ 
	\hline
	\printfreqnr
	& Ein Verwalter \textbf{MUSS} neue Benutzerkonten anlegen können. 
	& - \\
	\hline
	\printfreqnr
	& Der Verwalter \textbf{MUSS} dabei alle Informationen zum Benutzer (Anmelde- und Anzeigename, E-Mail-Adresse, Rolle + weitere Rollen-spezifische Angaben) in einer Eingabemaske erfassen können.
	& - \\ 
	\hline
	\printfreqnr
	& Für einen Studenten \textbf{MÜSSEN} die belegten Module angegeben werden können.
	& - \\ 
	\hline
	\printfreqnr
	& Für einen Dozenten \textbf{MÜSSEN} die Veranstaltungen angegeben werden können, die der neue Dozent übernehmen soll.
	& - \\ 
	\hline
\end{tabular}

\newpage

\begin{tabular} {|p{1,4cm}|p{10,7cm}|p{2,7cm}|}
	\hline
	ID & Anforderung & Kriterien \\
	\hline
	\printfreqnr
	& Das System \textbf{SOLL} bei der Erstellung von Benutzerkonten eine SSL-verschlüsselte Verbindung verwenden, um die Sicherheit der Daten zu gewährleisten.
	& - \\ 
	\hline
	\printfreqnr
	& Das System \textbf{MUSS} Verwaltern die Aktualisierung von Benutzerinformationen wie Name, E-Mail-Adresse und Rolle ermöglichen.
	& - \\ 
	\hline
	\printfreqnr
	& Die Eingabemaske zum Aktualisieren der Benutzerdaten \textbf{SOLL} Verwaltern die bisherigen Daten anzeigen.
	& - \\ 
	\hline
	\printfreqnr
	& Das System \textbf{SOLL} eine benutzerfreundliche Schnittstelle zur Aktualisierung von Benutzerinformationen bieten, die ohne spezielle Schulung bedienbar ist.
	& - \\ 
	\hline
	\printfreqnr
	& Wenn ein Verwalter ein Benutzerkonto löschen möchte, \textbf{SOLL} das System eine Bestätigung einholen, um unbeabsichtigte Löschungen zu vermeiden.
	& - \\ 
	\hline
	\printfreqnr
	& Das System \textbf{MUSS} die Bestätigungsanforderung klar und verständlich darstellen, um Fehlbedienungen zu minimieren.
	& - \\ 
	\hline
	\printfreqnr
	& Das System \textbf{MUSS} allen Akteuren außer dem Gast  ermöglichen, Passwörter für Benutzerkonten zurückzusetzen. 
	& - \\
	\hline
	\printfreqnr
	& Das System \textbf{SOLL} bei der Passwortrücksetzung starke Passwörter gemäß den aktuellen Sicherheitsstandards erzwingen. 
	& - \\
	\hline
\end{tabular}

\newpage

\subsubsection{Mögliche Änderungen}
Hier sind die möglichen Änderungen für die Anforderungen in diesem Unterkapitel.

\vspace{6pt}

\begin{tabular} {|p{4cm}|p{2,1cm}|p{8,7cm}|}
	\hline
	Anforderungen & Stabilität & Begründung \\
	\hline
	FA-701, FA-702
	& 100\%
	& Diese Anforderung ist Ausdruck der grundlegenden Designentscheidung, Termine zentral über die Verwalter betreuen zu lassen. \\
	\hline
	FA-703, FA-709
	& 100\%
	& Grundlegende Sicherheitsanforderung, die stabil bleibt. \\
	\hline
	FA-704, FA-705, FA-706, FA-707, FA-708, FA-710, FA-711, FA-715
	& 100\%
	& Grundfunktionalität, die stabil bleibt. \\
	\hline
	FA-712, FA-714
	& 100\%
	& UX-Funktionalität, die stabil bleibt \\
	\hline
	FA-713
	& 100\%
	& Anforderung, um versehentliche Löschoperationen zu vermeiden. \\
	\hline
	FA-716
	& 100\%
	& Grundlegende Sicherheitsanforderung, welche aktuelle Standards abbildet \\
	\hline
\end{tabular}

\newpage
%\vspace{18pt}

\subsection{Veranstaltungen verwalten}
Eine Veranstaltung ist in unserem Modell ein Muster für die Konkretisierung eines Moduls, welches seinerseits dann durch die Termine ausgekleidet wird. Das Verwalten von Veranstaltungen 
bedeutet das Anlegen, Ansehen, Bearbeiten oder Löschen von Veranstaltungsdatensätzen.\\
Da die Veranstaltungen gerade als Muster fungieren sollen,  werden nicht die Veranstaltungssätze, sondern die darauf fußenden Termindatensätze genutzt, um den Kalender u.Ä. darzustellen. Insofern beschränkt sich der Zugriff auf die Veranstaltungen auf die Verwalter.

\vspace{12pt}

\subsubsection{Angemeldete Verwalter }
Die Änderungen an den Veranstaltungen laufen gebündelt über die Verwaltung.

\vspace{12pt}

\begin{tabular} {|p{1,4cm}|p{10,7cm}|p{2,7cm}|}
	\hline
	ID & Anforderung & Kriterien \\
	\hline
	\printfreqnr
	& Ein angemeldeter Verwalter \textbf{MUSS} Veranstaltungen erstellen, löschen, bearbeiten und auslesen können.
	& - \\
	\hline
	\printfreqnr
	& Dem angemeldeten Verwalter \textbf{SOLL} das Erstellen einer Veranstaltung für ein Semester aus der Modulliste ermöglicht werden.
	& - \\
	\hline
	\printfreqnr
	& Der angemeldete Verwalter \textbf{SOLL} die erstellten Veranstaltung und ihre Termine in einer Liste sehen können. 
	& - \\
	\hline
	\printfreqnr
	& Der angemeldete Verwalter \textbf{SOLL} eine Veranstaltung über die Kalendersicht mit Terminen befüllen können.
	& - \\
	\hline
	\printfreqnr
	& Der angemeldete Verwalter \textbf{SOLL} beim Erstellen einer neuen Veranstaltung den zugehörigen Dozenten aus einem Dropdown-Menü auswählen können.
	& - \\
	\hline
	\printfreqnr
	& Der angemeldete Verwalter \textbf{SOLL} beim Erstellen einer neuen Veranstaltung den gewünschten Veranstaltungstyp aus einem Dropdown-Menü auswählen können.
	& - \\
	\hline
\end{tabular}

\newpage

\subsubsection{Weitere Akteure}
Da die Veranstaltungen lediglich als Muster dienen, haben die anderen Akteure keinen Zugriff auf diese. Studenten können ihre Termine auch ohne Zugriff auf die Veranstaltungen einsehen während Dozenten Änderungswünsche einfach der Verwaltung mitteilen können. Dies hat den Hintergrund, dass die Bearbeitung von Vorlagen gebündelt stattfinden sollte.

\vspace{12pt}

\begin{tabular} {|p{1,4cm}|p{10,7cm}|p{2,7cm}|}
	\hline
	ID & Anforderung & Kriterien \\
	\hline
	\printfreqnr
	& Das System \textbf{MUSS} dem Gast, dem Studenten und dem Dozenten den Zugriff auf die Veranstaltungsdatensätze verwehren.
	& - \\
	\hline
\end{tabular}

\newpage
%\vspace{12px}

\subsubsection{Mögliche Änderungen}
Hier sind die möglichen Änderungen für die Anforderungen in diesem Unterkapitel.

\vspace{6pt}

\begin{tabular} {|p{4cm}|p{2,1cm}|p{8,7cm}|}
	\hline
	Anforderungen & Stabilität & Begründung \\
	\hline
	FA-801
	& 100\%
	& Grundfunktionalität, die stabil bleibt. \\
	\hline
	FA-802, FA-803, FA-804, FA-805
	& 85\%
	& Praktisch stabil, könnte jedoch durch Änderungen in der Benutzeroberfläche beeinflusst werden. \\
	\hline
	FA-806
	& 90\%
	& Sicherheitsanforderung, die stabil bleibt. \\
	\hline
	FA-807
	& 100\%
	& Grundlegende Sicherheitsanforderung, die stabil bleibt. \\
	\hline
\end{tabular}

\newpage

\subsection{Module verwalten}
\begin{tabular} {|p{1,4cm}|p{10,7cm}|p{2,7cm}|}
	\hline
	ID & Anforderung & Kriterien \\
	\hline
	\printfreqnr
	& Das System \textbf{MUSS} dem Verwalter ermöglichen, neue Module mit relevanten Informationen wie Modulnamen, Beschreibungen und zugehörigen Veranstaltungen zu erstellen. 
	& - \\
	\hline
	\printfreqnr
	& Das System \textbf{MUSS} sicherstellen, dass bei der Erstellung von Modulen alle notwendigen Informationen validiert werden. 
	& - \\
	\hline
	\printfreqnr
	& Das System \textbf{MUSS} dem Verwalter ermöglichen, bestehende Modulinformationen zu aktualisieren. 
	& - \\
	\hline
	\printfreqnr
	& Das System \textbf{SOLL} sicherstellen, dass Änderungen an Modulen keine negativen Auswirkungen auf die Kurszuordnung oder Stundenpläne haben. 
	& - \\
	\hline
	\printfreqnr
	& Das System \textbf{MUSS} das Entfernen von Modulen aus der Datenbank durch den Verwalter unterstützen. 
	& - \\
	\hline
	\printfreqnr
	& Das System \textbf{MUSS} eine Bestätigung vom Benutzer einholen, bevor ein Modul endgültig entfernt wird, um unbeabsichtigte Löschungen zu verhindern. 
	& - \\
	\hline
	\printfreqnr
	& Das System \textbf{MUSS} sicherstellen, dass beim Entfernen von Modulen alle abhängigen Daten konsistent und fehlerfrei aktualisiert werden. 
	& - \\
	\hline
\end{tabular}

\newpage

\begin{tabular} {|p{1,4cm}|p{10,7cm}|p{2,7cm}|}
	\hline
	ID & Anforderung & Kriterien \\
	\hline
	\printfreqnr
	& Das System \textbf{MUSS} den Akteuren eine Übersicht aller Module zur Verfügung stellen, einschließlich der Informationen zu beteiligten Dozenten und Veranstaltungen. 
	& - \\
	\hline
	\printfreqnr
	& Das System \textbf{SOLL} für die Modulübersicht anpassbare Ansichten unterstützen, die es dem Benutzer ermöglichen, die Daten nach verschiedenen Kriterien zu sortieren und zu filtern. 
	& - \\
	\hline
	\printfreqnr
	& Das System \textbf{SOLL} auf die entsprechenden Stellen im Modulhandbuch verweisen. 
	& - \\
	\hline
	\printfreqnr
	& Das System \textbf{MUSS} die Integration der Module in die Stundenpläne der Studenten und Dozenten sicherstellen. 
	& - \\
	\hline
	\printfreqnr
	& Das System \textbf{MUSS} die nahtlose Integration neuer und aktualisierter Module in bestehende Stundenpläne ermöglichen, ohne dass es zu Konflikten oder Fehlern kommt. 
	& - \\
	\hline
	\printfreqnr
	& Das System \textbf{SOLL} eine Schnittstelle bieten, die es ermöglicht, Module direkt aus der Planungsansicht heraus zu verwalten. 
	& - \\
	\hline
	\printfreqnr
	& Das System \textbf{MUSS} dafür sorgen, dass die Verfügbarkeit der Stundenplan- und Moduldaten jederzeit gewährleistet ist, um kontinuierlichen Zugriff für die Benutzer zu bieten. 
	& - \\
	\hline
\end{tabular}

\newpage

\subsubsection{Mögliche Änderungen}
Hier sind die möglichen Änderungen für die Anforderungen in diesem Unterkapitel.

\vspace{6pt}

\begin{tabular} {|p{4cm}|p{2,1cm}|p{8,7cm}|}
	\hline
	Anforderungen & Stabilität & Begründung \\
	\hline
	FA-901, FA-903, FA-905, FA-907, FA-908, FA-914
	& 100\%
	& Grundlegende Anforderung, die stabil bleibt. \\
	\hline
	FA-902, FA-904
	& 100\%
	& Grundlegende Anforderung zur Datenintegrität. \\
	\hline
	FA-906
	& 100\%
	& Anforderung, um versehentliche Löschoperationen zu vermeiden. \\
	\hline
	FA-909
	& 75\%
	& UX-Anforderung, bei der sich gegebenenfalls die Filterlogik und die filterbaren Parameter ändern. \\
	\hline
	FA-910
	& 100\%
	& Einfache  Anforderung, die stabil bleibt. \\
	\hline
	FA-911, FA-912
	& 85\%
	& Integritätsanforderung, welche stabil bleibt, könnte jedoch durch Änderungen in der Systemlogik beeinflusst werden. \\
	\hline
	FA-913
	& 80\%
	& Praktisch stabil, könnte jedoch durch Änderungen in der Benutzeroberfläche beeinflusst werden. \\
	\hline
\end{tabular}

