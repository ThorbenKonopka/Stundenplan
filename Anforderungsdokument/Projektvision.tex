% !TeX root = Anforderungsdokument.tex

\clearpage

\section{Projektvision}
In unserem Projekt geht es darum, ein System zu planen und zu implementieren, mit dem das ITC ihre Termine verwalten kann. Dafür werden ein Ziel, ein Zweck und der Nutzen des Projektes bestimmt und ausformuliert.

\vspace{18pt}

\subsection{Ziel}
Unser Ziel ist es, für das IT-Center Dortmund eine Softwarelösung für das Anlegen und die Organisation von Stundenplänen zu erstellen. 

\vspace{18pt}

\subsection{Zweck}
Das Anlegen von Terminen ist in Bezug auf die Frage, wann ein Termin stattfinden soll, nicht einfach zu beantworten. Sachen wie die zwei Tage pro Woche im Unternehmen in den ersten vier Semestern, die Arbeitsmoral von den Studierenden zu bestimmten Uhrzeiten und nicht wenig bereits existierende Termine machen die Terminfindung zu einer komplexen Angelegenheit. Zudem gibt es gelegentlich Probleme mit Terminen, weil der Stundenplan nicht immer korrekt ist. Durch unsere Softwarelösung sollen dieser Prozess vereinfacht und fehlerhafte Termine vermieden werden.

\vspace{18pt}

\subsection{Nutzen}
Unsere Softwarelösung soll die Planung der Termine stark vereinfachen. Dadurch kann sich das ITC in Zukunft erhoffen, weniger Diskussionsgrundlagen mit den Partnerunternehmen zu schaffen und potenzielle Unzufriedenheit zu vermeiden. Zusätzlich könnte Zeit für die Planung der Termine minimiert werden, was die Arbeitslast des IT-Centers reduzieren könnte.