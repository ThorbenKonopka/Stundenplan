% !TeX root = Anforderungsdokument.tex

\clearpage

\section{Aufwände schätzen}
Schätzung in pt.

\vspace{18pt}

\subsection{Einloggen}
Wir finden, dass eine einfache Anmeldung mit Benutzername und Passwort völlig ausreicht. Es macht zudem keinen Sinn verschiedene Anmeldefenster für die Verschiedenen Benutzerarten zu erstellen. Es ist vorgesehen, dass anhand des Benutzernamens herausgefunden werden kann, was für ein Benutzer sich gerade angemeldet hat.

\vspace{12pt}

\begin{tabular} {|p{1,1cm}|p{11cm}|p{2,7cm}|}
	\hline
	ID & Anforderung & Kriterien \\
	\hline
	FA-
	& Wenn ein unangemeldeter Benutzer das System aufruft \textbf{MUSS} Das System den Benutzer dazu auffordern sich mit seinem Benutzernamen und seinem Password anzumelden.
	& - \\
	\hline
	FA- 
	& Wenn ein unangemeldeter Benutzer einen richtigen Benutzernamen und das richtige Passwort eingegeben hat \textbf{MUSS} das System den Benutzer anmelden und weiterleiten.
	& -  \\
	\hline
	FA- 
	& Wenn ein unangemeldeter Benutzer einen falschen Benutzernamen oder ein falsches Passwort eingegeben hat, \textbf{MUSS} das System den Benutzer darauf hinweisen, dass die Anmeldung nicht erfolgreich war.
	& - \\
	\hline
	FA-
	& Wenn eine Anmeldung nicht erfolgreich war, \textbf{DARF} das System \textbf{NICHT} anzeigen, welche Eingabe nicht korrekt war.
	& - \\
	\hline
\end{tabular}
Min: 1pt
Max: 2,5pt
Schätzung: 2pt

\newpage

\subsection{Termine einsehen}
Dieses System ist in erster Linie dazu da um Termine zu verwalten. Bei der Betrachtung dieser Funktionalität spielen die verschiedenen Benutzergruppen bzw. die verschiedenen Sichten auf diese Funktionalitäten eine große Rolle. 

\begin{tabular} {|p{1,1cm}|p{11cm}|p{2,7cm}|}
	\hline
	ID & Anforderung & Kriterien \\
	\hline
	FA-
	& Wenn ein unangemeldeter Benutzer das System aufruft \textbf{SOLL} das System dem Benutzer die Möglichkeit geben alle Termine einzusehen. 
	& NFA-2, NFA-3 \\
	\hline
	FA- 
	& Wenn ein unangemeldeter Benutzer alle Termine einsieht \textbf{SOLL} das System dem Benutzer die Möglichkeit geben Jahrgänge auszuwählen.
	& -  \\
	\hline
	FA- 
	& Wenn ein unangemeldeter Benutzer alle Termine einsieht \textbf{SOLL} das System dem Benutzer die Möglichkeit geben die Gruppen \texttt{A-F} und \texttt{G-L} auszuwählen.
	& - \\
	\hline
	FA-
	& Wenn ein unangemeldeter Benutzer einen oder mehrere Jahrgänge ausgewählt hat, \textbf{MUSS} das System alle Termine anzeigen, die sich den ausgewählten Jahrgängen zuordnen lassen.
	& - \\
	\hline
	FA-
	& Wenn ein unangemeldeter Benutzer eine oder alle Gruppen ausgewählt hat \textbf{MUSS} das System alle Termine anzeigen, die sich den ausgewählten Gruppen zuordnen lassen.
	& - \\
	\hline
	FA-
	& Wenn ein unangemeldeter Benutzer sowohl Jahrgänge als auch Gruppen ausgewählt hat \textbf{MUSS} das System die Schnittmenge der beiden Termin-Menge anzeigen.
	& - \\
	\hline
	FA-
	& Wenn ein unangemeldeter Benutzer keine Gruppe ausgewählt hat \textbf{MUSS} das System diesen Filter ignorieren.
	& - \\
	\hline
	FA-
	& Wenn ein unangemeldeter Benutzer keine Jahrgänge ausgewählt hat \textbf{MUSS} das System diesen Filter ignorieren.
	& - \\
	\hline
\end{tabular}
Min: 4,5pt
Max: 7pt
Schätzung: 6pt

\newpage

\begin{tabular} {|p{1,1cm}|p{11cm}|p{2,7cm}|}
	\hline
	ID & Anforderung & Kriterien \\
	\hline
	FA-
	& Wenn ein Angemeldeter Student oder Dozent seine Termine einsehen möchte \textbf{MUSS} das System die Termine anzeigen, die sich dem Studenten oder Dozenten zuordnen lassen. 
	& - \\
	\hline
\end{tabular}

\newpage

\begin{tabular} {|p{1,1cm}|p{11cm}|p{2,7cm}|}
	\hline
	ID & Anforderung & Kriterien \\
	\hline
	FA-
	& Wenn ein angemeldeter Verwalter alle Termine einsehen möchte \textbf{MUSS} das System alle Termine anzeigen. 
	& - \\
	\hline
	FA-
	& Wenn ein angemeldeter Verwalter alle Termine angezeigt bekommt \textbf{MUSS} dem Benutzer die Möglichkeit in einem Suchfeld eine Matrikelnummer oder einen Namen einzugeben. 
	& - \\
	\hline
	FA-
	& Wenn ein angemeldeter Verwalter in dem Suchfeld mindestens einen Buchstaben eingegeben hat, \textbf{MUSS} das System mögliche Studenten, bei denen die eingegebenen Zeichen entweder bei der Matrikelnummer oder bei dem Namen übereinstimmen, auflisten und zur Auswahl bereitstellen. 
	& - \\
	\hline
	FA-
	& Wenn ein angemeldeter Verwalter einen Studenten ausgewählt hat \textbf{MUSS} das System alle Termine des ausgewählten Studenten anzeigen. 
	& - \\
	\hline
	FA-
	& Wenn ein angemeldeter Verwalter einen Studenten ausgewählt hat \textbf{MUSS} das System dem Verwalter die Möglichkeit geben, den Studenten als Filter zu entfernen. 
	& - \\
	\hline
	FA-
	& Wenn ein angemeldeter Verwalter einen Studenten ausgewählt hat \textbf{DARF} das System \textbf{NICHT} dem Verwalter die Möglichkeit geben, einen weiteren Studenten auszuwählen. 
	& - \\
	\hline
\end{tabular}

\newpage

\begin{tabular} {|p{1,1cm}|p{11cm}|p{2,7cm}|}
	\hline
	ID & Anforderung & Kriterien \\
	\hline
	FA-
	& Wenn ein unangemeldeter Benutzer alle Termine angezeigt bekommt \textbf{MUSS} das System, das Datum, die Start- und Endzeit, der Dozent, die Bezeichnung des Moduls und falls vorhanden die Raumnummer anzeigen. 
	& - \\
	\hline
	FA-
	& Wenn ein angemeldeter Dozent alle Termine angezeigt bekommt \textbf{MUSS} das System, das Datum, die Start- und Endzeit, die Bezeichnung des Moduls, falls vorhanden die Raumnummer und falls vorhanden der Link zu einer Onlineplattform anzeigen. 
	& - \\
	\hline
	FA-
	& Wenn ein angemeldeter Student alle Termine angezeigt bekommt \textbf{MUSS} das System, das Datum, die Start- und Endzeit, der Dozent, die Bezeichnung des Moduls, falls vorhanden die Raumnummer und falls vorhanden der Link zu einer Onlineplattform anzeigen. 
	& - \\
	\hline
	FA-
	& Wenn ein angemeldeter Student alle Termine angezeigt bekommt \textbf{MUSS} das System , das Datum, die Start- und Endzeit, der Dozent, die Bezeichnung des Moduls, falls vorhanden die Raumnummer und falls vorhanden der Link zu einer Onlineplattform anzeigen.
	& - \\
	\hline
	FA-
	& Das System \textbf{DARF NICHT} den Link für Onlinevorlesungen für unangemeldete Benutzer anzeigen. 
	& - \\
	\hline
\end{tabular}
Min: 1pt
Max: 2,5pt
Schätzung: 2pt

\newpage

\subsection{Termine verwalten}
Unter Termine verwalten versteht man, die Bearbeitung, die Erstellung und das Löschen von Terminen. Dabei haben verschiedene Benutzergruppen verschiedene Anforderungen an die Verwaltung von Terminen.

\begin{tabular} {|p{1,1cm}|p{11cm}|p{2,7cm}|}
	\hline
	ID & Anforderung & Kriterien \\
	\hline
	FA-
	& Das System \textbf{DARF NICHT} erlauben, dass Studenten oder unangemeldete Benutzer, Termine in irgendeiner Form verwalten. 
	& - \\
	\hline
	FA-
	& Wenn ein angemeldeter Student oder ein unangemeldeter Benutzer versucht einen Termine zu verwalten (bearbeiten, erstellen oder löschen), dann \textbf{MUSS} das System weiteres Vorgehen unterbinden und dem Benutzer mitteilen, dass er nicht autorisiert ist.
	& - \\ 
	\hline
\end{tabular}

%\vspace{12pt}
\newpage

\begin{tabular} {|p{1,1cm}|p{11cm}|p{2,7cm}|}
	\hline
	ID & Anforderung & Kriterien \\
	\hline
	FA-
	& Wenn ein angemeldeter Dozent Termine anlegen möchte \textbf{MUSS} das System einen neuen Vorschlag erstellen und diesen von dem Dozenten bearbeiten lassen.
	& - \\
	\hline
	FA-
	& Wenn ein angemeldeter Dozent einen Vorschlag bearbeitet, \textbf{MUSS} das System den Benutzer dazu auffordern \textbf{einen} Jahrgang auszuwählen. 
	& - \\
	\hline
	FA-
	& Wenn ein angemeldeter Dozent einen Vorschlag bearbeitet und ein Jahrgang ausgewählt wurde, \textbf{MUSS} das System dem Benutzer die Möglichkeit geben einen neuen Termin anzulegen.  
	& - \\
	\hline
	FA-
	& Wenn ein angemeldeter Dozent einen neuen Termin anlegt, \textbf{MUSS} das System den Benutzer fragen, für welche Veranstaltung der Termin gedacht ist.
	& - \\
	\hline
	FA-
	& Wenn ein angemeldeter Dozent einen neuen Termin anlegt und die Veranstaltung ausgewählt hat, \textbf{MUSS} das System den Benutzer die Möglichkeit geben, die restlichen Daten (Start- und Endzeit, Raumnummer, Link für die Vorlesung) einzutragen. 
	& - \\
	\hline
\end{tabular}

\begin{tabular} {|p{1,1cm}|p{11cm}|p{2,7cm}|}
	\hline
	ID & Anforderung & Kriterien \\
	\hline
	FA-
	& Wenn ein angemeldeter Dozent alle benötigten Daten eingegeben hat, \textbf{MUSS} das System verifizieren, dass der neue Termin in einem freien Zeitraum liegt.
	& - \\
	\hline
	FA-
	& Wenn das System verifiziert hat, dass der Termin in einem freien Zeitraum liegt, \textbf{MUSS} das System versuchen den Termin in dem Vorschlag zu speichern.
	& - \\
	\hline
	FA-
	& Das System feststellt, dass der Termin in einem belegten Zeitraum liegt, \textbf{MUSS} das System weiteres Vorgehen unterbinden und den Dozenten darauf hinweisen, dass der Termin in diesem Zeitraum nicht angelegt werden kann.
	& - \\
	\hline
	FA-
	& Wenn ein angemeldeter Dozent mindestens einen Termin angelegt hat, \textbf{MUSS} das System dem Dozenten die Möglichkeit geben den Vorschlag freizugeben, damit ein Verwalter diesen einsehen kann.
	& - \\
	\hline
	FA-
	& Ein Dozent \textbf{MUSS} einen seiner angelegten Termine im Nachhinein bezüglich der Raumnummer oder des Links zur Online-Veranstaltung selbst ändern können.
	& - \\
	\hline
\end{tabular}

\newpage

\begin{tabular} {|p{1,1cm}|p{11cm}|p{2,7cm}|}
	\hline
	ID & Anforderung & Kriterien \\
	\hline
	FA-
	& Wenn ein angemeldeter Dozent einen Vorschlag bearbeitet, \textbf{MUSS} das System dem Benutzer auf KI-basierende Termine Vorschlagen. 
	& - \\
	\hline
	FA-
	& Wenn ein angemeldeter Dozent einen auf KI-basierenden Termin auswählt \textbf{MUSS} das System die gleichen Daten, wie bei dem Normalen anlegen fragen. Die einzige Ausnahme ist der Zeitraum. Dieser soll vor befüllt aber bearbeitbar sein. 
	& - \\
	\hline
\end{tabular}

\begin{tabular} {|p{1,1cm}|p{11cm}|p{2,7cm}|}
	\hline
	ID & Anforderung & Kriterien \\
	\hline
	FA-
	& Wenn ein angemeldeter Dozent einen Vorschlag eingereicht hat, \textbf{MUSS} das System dem angemeldeten Verwalter den Vorschlag anzeigen und Ihm die Möglichkeit geben den Vorschlag anzuwenden. 
	& - \\
	\hline
	FA-
	& Wenn der angemeldeter Verwalter den Vorschlag anwendet, \textbf{MUSS} das System versuchen die enthaltenen Termine zu persistieren.
	& - \\
	\hline
	FA-
	& Wenn der angemeldeter Verwalter den Vorschlag anwendet, \textbf{MUSS} das System sicherstellen, dass es keine Überlappungen mit anderen Terminen gibt.
	& - \\
	\hline
	FA-
	& Wenn der angemeldeter Verwalter den Vorschlag ablehnt, \textbf{MUSS} das System den Verwalter dazu auffordern einen Grund anzugeben und den betroffenen Dozenten benachrichtigen.
	& - \\
	\hline
	FA-
	& Wenn der angemeldeter Verwalter einen neuen Termin anlegt, \textbf{MUSS} das System sicher stellen, dass es keine Überlappungen mit anderen Terminen bei beteiligten Studenten gibt.
	& - \\
	\hline
	FA-
	& Wenn es keine Überlappungen geben sollte, \textbf{MUSS} das System versuchen den Termin anzulegen.
	& - \\
	\hline
	FA-
	& Wenn es Überlappungen geben sollte, \textbf{MUSS} das System den Verwalter dazu auffordern einen anderen Zeitraum zu wählen.
	& - \\
	\hline
	FA-
	& Ein Verwalter \textbf{MUSS} ebenfalls die Möglichkeit haben Termine über einen Vorschlag mit dem Dozenten abzustimmen.
	& - \\
	\hline
	FA-
	& Wenn ein Verwalter versucht einen Termin zu verändern oder zu löschen \textbf{MUSS} das System versuchen den Termin zu verändern oder zu löschen.
	& - \\
	\hline
	FA-
	& Wenn ein Verwalter versucht einen Termin zu verändern \textbf{MUSS} das System sicher stellen, dass es keine Überlappungen mit anderen Terminen gibt.
	& - \\
	\hline
\end{tabular}
Min: 4pt
Max: 6pt
Schätzung: 4,5pt

\newpage

\subsection{Termine verschieben}

\begin{tabular} {|p{1,1cm}|p{11cm}|p{2,7cm}|}
	\hline
	ID & Anforderung & Kriterien \\
	\hline
	FA-
	& Das System \textbf{DARF NICHT} erlauben, dass Studenten oder unangemeldete Benutzer, Termine in irgendeiner Form verschieben. 
	& - \\
	\hline
	FA-
	& Wenn ein angemeldeter Student oder ein unangemeldeter Benutzer versucht einen Termine zu verschieben, dann \textbf{MUSS} das System weiteres Vorgehen unterbinden und dem Benutzer mitteilen, dass er nicht autorisiert ist.
	& - \\ 
	\hline
\end{tabular}

\begin{tabular} {|p{1,1cm}|p{11cm}|p{2,7cm}|}
	\hline
	ID & Anforderung & Kriterien \\
	\hline
	FA-
	& Ein Dozent \textbf{MUSS} die Uhrzeit eines Termins ändern können.
	& - \\
	\hline
	FA-
	& Ein Dozent \textbf{DARF NICHT} selbst das Datum eines Termins ändern.
	& - \\
	\hline
	FA-
	& Ein Dozent \textbf{MUSS} eine Anfrage zum Ändern des Datums eines Termins an die Verwaltung stellen können.
	& - \\
	\hline
\end{tabular}

\begin{tabular} {|p{1,1cm}|p{11cm}|p{2,7cm}|}
	\hline
	ID & Anforderung & Kriterien \\
	\hline
	FA-
	& Ein Verwalter \textbf{MUSS} einen Termin in seiner Uhrzeit und seinem Datum verschieben können.
	& - \\
	\hline
	FA-
	& Ein Verwalter \textbf{MUSS} die gestellten Datumsänderungsanfragen von Dozenten einsehen können.
	& - \\
	\hline
	FA-
	& Ein Verwalter \textbf{MUSS} die Verschiebe-Anfragen von Dozenten annehmen oder ablehnen können, je nachdem, ob sich diese mit anderen Terminen überschneiden oder durch den neuen Termin Richtlinien verletzt werden würden.
	& - \\
	\hline
	FA-
	& Ein Verwalter \textbf{MUSS} die durchgeführten Uhrzeitänderungen von Dozenten einsehen können und diese rückgängig machen, falls diese zu Konflikten führen.
	& - \\
	\hline
\end{tabular}
Min: 4pt
Max: 6pt
Schätzung: 5pt

\newpage

\subsection{Anwesenheitsliste verwalten}

\begin{tabular} {|p{1,1cm}|p{11cm}|p{2,7cm}|}
	\hline
	ID & Anforderung & Kriterien \\
	\hline
	FA-
	& Das System \textbf{DARF NICHT} erlauben, dass Studenten oder unangemeldete Benutzer, Anwesenheitslisten in irgendeiner Form verwalten oder einsehen. 
	& - \\
	\hline
	FA-
	& Wenn ein angemeldeter Student oder ein unangemeldeter Benutzer versucht eine Anwesenheitsliste zu verwalten oder einzusehen, dann \textbf{MUSS} das System weiteres Vorgehen unterbinden und dem Benutzer mitteilen, dass er nicht autorisiert ist.
	& - \\ 
	\hline
\end{tabular}
Min: 3pt
Max: 5,5pt
Schätzung: 4pt

\newpage

\subsection{Fehlzeiten einsehen}

\begin{tabular} {|p{1,1cm}|p{11cm}|p{2,7cm}|}
	\hline
	ID & Anforderung & Kriterien \\
	\hline
	FA-
	& Das System \textbf{DARF NICHT} erlauben, dass unangemeldete Benutzer, Fehlzeiten in irgendeiner Form einsehen. 
	& - \\
	\hline
	FA-
	& Wenn ein unangemeldeter Benutzer versucht einen Fehlzeiten einzusehen, \textbf{MUSS} das System weiteres Vorgehen unterbinden und dem Benutzer mitteilen, dass er nicht autorisiert ist.
	& - \\ 
	\hline
\end{tabular}

\begin{tabular} {|p{1,1cm}|p{11cm}|p{2,7cm}|}
	\hline
	ID & Anforderung & Kriterien \\
	\hline
	FA-
	& Ein Student \textbf{MUSS} seine Fehlzeiten einsehen können. 
	& - \\
	\hline
	FA-
	& Die angezeigten Fehlzeiten \textbf{MÜSSEN} notwendige Informationen über den Termin enthalten.
	& - \\ 
	\hline
\end{tabular}

\begin{tabular} {|p{1,1cm}|p{11cm}|p{2,7cm}|}
	\hline
	ID & Anforderung & Kriterien \\
	\hline
	FA-
	& Ein Dozent \textbf{DARF NICHT} Fehlzeiten einsehen. Hier ist die Anwesenheitsliste eine Ausnahme. 
	& - \\
	\hline
\end{tabular}

\begin{tabular} {|p{1,1cm}|p{11cm}|p{2,7cm}|}
	\hline
	ID & Anforderung & Kriterien \\
	\hline
	FA-
	& Ein Verwalter \textbf{MUSS} alle Fehlzeiten einsehen können. 
	& - \\
	\hline
	FA-
	& Ein Verwalter \textbf{MUSS} alle Fehlzeiten zu einem spezifizierten Studenten in Erfahrung bringen können.
	& - \\ 
	\hline
	FA-
	& Wenn ein Verwalter anfängt die Matrikelnummer oder den Namen des Studenten einzugeben, \textbf{MUSS} das System passende Studenten vorschlagen.
	& - \\ 
	\hline
\end{tabular}
Min: 1,5pt
Max: 3pt
Schätzung: 2pt

\subsection{Benutzer verwalten}
\begin{tabular} {|p{1,1cm}|p{11cm}|p{2,7cm}|}
	\hline
	ID & Anforderung & Kriterien \\
	\hline
	FA-
	& Ein Verwalter \textbf{MUSS} neue Benutzerkonten anlegen können. 
	& - \\
	\hline
	FA-
	& Das System \textbf{MUSS} die Erstellung von Benutzerkonten innerhalb von 5 Sekunden nach Eingabe der erforderlichen Daten ermöglichen, um eine schnelle Einrichtung zu gewährleisten. 
	& - \\ 
	\hline
	FA-
	& Das System \textbf{SOLL} bei der Erstellung von Benutzerkonten eine SSL-verschlüsselte Verbindung verwenden, um die Sicherheit der Daten zu gewährleisten.
	& - \\ 
	\hline
	FA-
	& Das System \textbf{MUSS} Verwaltern die Aktualisierung von Benutzerinformationen wie Name, E-Mail-Adresse und Rolle ermöglichen.
	& - \\ 
	\hline
	FA-
	& Das System \textbf{MUSS} sicherstellen, dass Änderungen an Benutzerkonten entsprechend den Datenschutzrichtlinien protokolliert werden.
	& - \\ 
	\hline
	FA-
	& Das System \textbf{SOLL} eine benutzerfreundliche Schnittstelle zur Aktualisierung von Benutzerinformationen bieten, die ohne spezielle Schulung bedienbar ist.
	& - \\ 
	\hline
	FA-
	& Wenn ein Verwalter ein Benutzerkonto löschen möchte, \textbf{SOLL} das System eine Bestätigung einholen, um unbeabsichtigte Löschungen zu vermeiden.
	& - \\ 
	\hline
	FA-
	& Das System \textbf{MUSS} gelöschte Benutzerdaten gemäß den gesetzlichen Aufbewahrungsfristen speichern, bevor sie endgültig entfernt werden.
	& - \\ 
	\hline
	FA-
	& Das System \textbf{SOLL} regelmäßig Backups der Benutzerdaten erstellen, um Datenverlust bei einem Fehler zu verhindern.
	& - \\ 
	\hline
	FA-
	& Das System \textbf{MUSS} die Bestätigungsanforderung klar und verständlich darstellen, um Fehlbedienungen zu minimieren.
	& - \\ 
	\hline
\end{tabular}
Min: 4,5pt
Max: 6,5pt
Schätzung: 5pt

\newpage

\begin{tabular} {|p{1,1cm}|p{11cm}|p{2,7cm}|}
	\hline
	ID & Anforderung & Kriterien \\
	\hline
	FA-
	& Das System \textbf{MUSS} allen Akteuren außer dem Gast  ermöglichen, Passwörter für Benutzerkonten zurückzusetzen. 
	& - \\
	\hline
	FA-
	& Das System \textbf{MUSS} einen sicheren Prozess für das Zurücksetzen von Passwörtern bieten, einschließlich der Bestätigung durch eine Zwei-Faktor-Authentifizierung. 
	& - \\
	\hline
	FA-
	& Das System \textbf{SOLL} in der Lage sein, das Zurücksetzen von Passwörtern innerhalb von 60 Sekunden durchzuführen. 
	& - \\
	\hline
	FA-
	& Das System \textbf{SOLL} bei der Passwortrücksetzung starke Passwörter gemäß den aktuellen Sicherheitsstandards erzwingen. 
	& - \\
	\hline
\end{tabular}
Min: 4,5pt
Max: 6,5pt
Schätzung: 5pt

%\newpage
\vspace{18pt}

\subsection{Veranstaltungen verwalten}
\begin{tabular} {|p{1,1cm}|p{11cm}|p{2,7cm}|}
	\hline
	ID & Anforderung & Kriterien \\
	\hline
	FA-
	& Das System \textbf{SOLL} dem Verwalter ermöglichen, neue Veranstaltungen mit Details hinzuzufügen. 
	& - \\
	\hline
	FA-
	& Das System \textbf{MUSS} die Erstellung neuer Veranstaltungen innerhalb von maximal 10 Sekunden ermöglichen, um eine effiziente Verwaltung zu gewährleisten. 
	& - \\
	\hline
	FA-
	& Das System \textbf{MUSS} sicherstellen, dass Veranstaltungsdaten über sichere Protokolle (wie HTTPS) übertragen werden, um die Integrität und Vertraulichkeit der Informationen zu schützen. 
	& - \\
	\hline
	FA-
	& Das System \textbf{SOLL} in der Lage sein, eine unbegrenzte Anzahl von Veranstaltungen zu unterstützen, ohne an Leistung zu verlieren. 
	& - \\
	\hline
	FA-
	& Das System \textbf{MUSS} das Bearbeiten von Veranstaltungsdetails durch den Verwalter ermöglichen. 
	& - \\
	\hline
	FA-
	& Das System \textbf{MUSS} Änderungen an Veranstaltungsdetails sofort in allen abhängigen Systemen und Ansichten aktualisieren, um Konsistenz zu gewährleisten. 
	& - \\
	\hline
	FA-
	& Das System \textbf{MUSS} Benutzerschnittstellen bereitstellen, die intuitiv und ohne zusätzliche Schulung zu bedienen sind. 
	& - \\
	\hline
\end{tabular}

\newpage

\begin{tabular} {|p{1,1cm}|p{11cm}|p{2,7cm}|}
	\hline
	ID & Anforderung & Kriterien \\
	\hline
	FA-
	& Das System \textbf{SOLL} dem Verwalter ermöglichen, Veranstaltungen aus dem System zu entfernen. 
	& - \\
	\hline
	FA-
	& Das System \textbf{MUSS} eine Warnmeldung anzeigen und eine Bestätigung verlangen, bevor eine Veranstaltung endgültig gelöscht wird, um unbeabsichtigte Löschungen zu verhindern. 
	& - \\
	\hline
	FA-
	& Das System \textbf{SOLL} gelöschte Veranstaltungen in einem Archiv für einen definierten Zeitraum speichern, um eine mögliche Wiederherstellung zu ermöglichen. 
	& - \\
	\hline
	FA-
	& Das System \textbf{MUSS} sicherstellen, dass die Löschung von Veranstaltungen keine Auswirkungen auf die Gesamtstabilität des Systems hat. 
	& - \\
	\hline
	FA-
	& Das System \textbf{SOLL} eine Übersicht aller geplanten Veranstaltungen bieten, die jeder Akteur einsehen kann. 
	& - \\
	\hline
	FA-
	& Das System \textbf{MUSS} die Übersicht aller Veranstaltungen in Echtzeit darstellen, um aktuelle Informationen bereitzustellen. 
	& - \\
	\hline
	FA-
	& Das System \textbf{SOLL} anpassbare Filter für die Veranstaltungsansicht bereitstellen, um eine individuelle Ansicht zu ermöglichen. 
	& - \\
	\hline
	FA-
	& Das System \textbf{MUSS} die Übersichtsseite so gestalten, dass sie auch bei einer großen Anzahl von Einträgen performant bleibt. 
	& - \\
	\hline
\end{tabular}
Min: 3pt
Max: 5pt
Schätzung: 4pt

\newpage

\subsection{Module verwalten}
\begin{tabular} {|p{1,1cm}|p{11cm}|p{2,7cm}|}
	\hline
	ID & Anforderung & Kriterien \\
	\hline
	FA-
	& Das System \textbf{MUSS} dem Verwalter ermöglichen, neue Module mit relevanten Informationen wie Modulnamen, Beschreibungen und zugehörigen Veranstaltungen zu erstellen. 
	& - \\
	\hline
	FA-
	& Das System \textbf{MUSS} sicherstellen, dass bei der Erstellung von Modulen alle notwendigen Informationen validiert werden. 
	& - \\
	\hline
	FA-
	& Das System \textbf{MUSS} dem Verwalter ermöglichen, bestehende Modulinformationen zu aktualisieren. 
	& - \\
	\hline
	FA-
	& Das System \textbf{SOLL} eine Versionsgeschichte für jedes Modul führen, um Änderungen nachverfolgen und bei Bedarf frühere Versionen wiederherstellen zu können. 
	& - \\
	\hline
	FA-
	& Das System \textbf{SOLL} sicherstellen, dass Änderungen an Modulen keine negativen Auswirkungen auf die Kurszuordnung oder Stundenpläne haben. 
	& - \\
	\hline
	FA-
	& Das System \textbf{MUSS} das Entfernen von Modulen aus der Datenbank durch den Verwalter unterstützen. 
	& - \\
	\hline
	FA-
	& Das System \textbf{MUSS} eine Bestätigung vom Benutzer einholen, bevor ein Modul endgültig entfernt wird, um unbeabsichtigte Löschungen zu verhindern. 
	& - \\
	\hline
	FA-
	& Das System \textbf{SOLL} entfernte Module in einem Archiv speichern, um eine eventuelle Wiederherstellung zu ermöglichen. 
	& - \\
	\hline
	FA-
	& Das System \textbf{MUSS} sicherstellen, dass beim Entfernen von Modulen alle abhängigen Daten konsistent und fehlerfrei aktualisiert werden. 
	& - \\
	\hline
\end{tabular}

\newpage

\begin{tabular} {|p{1,1cm}|p{11cm}|p{2,7cm}|}
	\hline
	ID & Anforderung & Kriterien \\
	\hline
	FA-
	& Das System \textbf{MUSS} den Akteuren eine Übersicht aller Module zur Verfügung stellen, einschließlich der Informationen zu beteiligten Dozenten und Veranstaltungen. 
	& - \\
	\hline
	FA-
	& Das System \textbf{MUSS} die Modulübersicht in Echtzeit aktualisieren, um jederzeit den aktuellen Stand wiedergeben zu können. 
	& - \\
	\hline
	FA-
	& Das System \textbf{SOLL} für die Modulübersicht anpassbare Ansichten unterstützen, die es dem Benutzer ermöglichen, die Daten nach verschiedenen Kriterien zu sortieren und zu filtern. 
	& - \\
	\hline
	FA-
	& Das System \textbf{SOLL} auf die entsprechenden Stellen im Modulhandbuch verweisen. 
	& - \\
	\hline
	FA-
	& Das System \textbf{MUSS} die Integration der Module in die Stundenpläne der Studenten und Dozenten sicherstellen. 
	& - \\
	\hline
	FA-
	& Das System \textbf{MUSS} die nahtlose Integration neuer und aktualisierter Module in bestehende Stundenpläne ermöglichen, ohne dass es zu Konflikten oder Fehlern kommt. 
	& - \\
	\hline
	FA-
	& Das System \textbf{SOLL} eine Schnittstelle bieten, die es ermöglicht, Module direkt aus der Planungsansicht heraus zu verwalten. 
	& - \\
	\hline
	FA-
	& Das System \textbf{MUSS} dafür sorgen, dass die Verfügbarkeit der Stundenplan- und Moduldaten jederzeit gewährleistet ist, um kontinuierlichen Zugriff für die Benutzer zu bieten. 
	& - \\
	\hline
\end{tabular}
Min: 1,5pt
Max: 2,5pt
Schätzung: 2pt

\newpage

\subsection{Anforderungen an den Client}
Es ist wichtig, dass die Termine plattformunabhängig angezeigt und geändert werden können. Termine müssen recht kurzfristig angepasst und eingesehen werden können. Das wäre z.B.: mit einem Fat-Client nicht gegeben, weil man bei der Verwendung einer Konkreten Programmiersprache in der Regel an eine oder eine Gruppe von Plattformen gebunden ist. Um kurzfristige Änderungen gewährleisten zu können, muss es möglich sein, sich z.B.: über das Handy an dem System anzumelden. Studenten sehen häufig den Stundenplan über ihr Handy ein. Zudem soll es möglich sein, dass sich alle Benutzer auch über einen Tower-PC oder einen Laptop am System anmelden können, weil dort besonders die Verwaltung angenehmer ist. Deshalb muss das System alle Funktionalitäten über Weboberflächen bereitstellen.

\vspace{12pt}

\begin{tabular} {|p{1,1cm}|p{11cm}|p{2,7cm}|}
	\hline
	ID & Anforderung & Kriterien \\
	\hline
	NFA-
	& Das System \textbf{MUSS} für alle Funktionalitäten eine Weboberfläche bereitstellen. 
	& NFA-200, NFA-300 \\
	\hline
	NFA-200 
	& Die Weboberflächen \textbf{SOLLEN} auf verschiedenen Bildschirmgrößen gut aussehen.
	& -  \\
	\hline
	NFA-300
	& Die Weboberflächen \textbf{MÜSSEN} auf verschiedenen Bildschirmgrößen bedienbar sein.
	& - \\
	\hline
	NFA-
	& Die Client-Webseite \textbf{SOLL} auch bei einer langsameren Internetverbindung in einer akzeptablen Zeitspanne geladen werden.
	& - \\
	\hline
\end{tabular}
Min: 7,5pt
Max: 10pt
Schätzung: 9pt

\newpage

\subsection{Anforderungen an den Server}
Es würde unserer Meinung nach Sinn machen, wenn man für die Funktionalitäten, die von dem Server bereitgestellt werden Schnittstellen definieren und dokumentieren würde. Dadurch könnten die Studenten selbst einen Client schreiben. Das könnte im Rahmen eines Moduls eine Aufgabe sein. 

\vspace{12pt}

\begin{tabular} {|p{1,1cm}|p{11cm}|p{2,7cm}|}
	\hline
	ID & Anforderung & Kriterien \\
	\hline
	NFA- 
	& Das System \textbf{SOLL} dokumentierte Schnittstellen für die Funktionalitäten bereitstellen.
	& - \\
	\hline
	NFA- 
	& Die Business-Logik des Servers \textbf{MUSS} in der Programmiersprache Java umgesetzt werden.
	& - \\
	\hline
\end{tabular}
Min: 11pt
Max: 15,5pt
Schätzung: 13pt
